\section{Лекция 6}
В прошлый раз мы разговаривали о дискретном преобразовании Фурье. Сегодня мы с вами разберём, что такое быстрое дискретное преобразование Фурье. Ускорение вычисления. Будем исследовать прямое преобразование Фурье. Нам надо вычислить 
\[
  A_k = \frac 1N \RY j1{N-1} f_j \exp \left( -2\pi i \frac{ kj}N \right),\quad k=0,\dots, N-1.
\]
У нас тут $N$ сумм по $N$ слагаемых. Сложность $N^2$. Можно ли ускорить этот процесс. Идея тут какая: мы попытаемся среди всех этих экспонент будем приводить подобные.

Мы рассмотрим случай, когда $N = p_1\cdot P_p$, причём $p_1,p_2\ne 1$. Тогда $k = k_1  + p_1 k_2$, $j = j_2 + p_2 j_1$ (нумерация такая себя оправдает позже). Здесь $0\le k,j\le p_1-1$, $0\le k_2,j_2\le p_2 -1$ (уже какой-то смысл нумерации просматривается).

Теперь берём эти разложения $k$ и $j$ подставим в эту сумму
\[
  A_k = \frac 1N \RY{j_1}0{p_1-1} \RY{j_2}0{p_2-1} f_{j_2 + p_2 j_1} \exp \left( 
   -2\pi i\frac{(k_1 + p_1 k_2)(j_2 + p_2 j_1)}{N}
 \right)
\]
Я попробую в аргументе экспоненты (поделённом на $-2\pi i$) выделить целую часть. Будем в сторонке вычислять вспомагательный результат, чтобы было понятно.
\[
  \frac{(k_1 + p_1 k_2)(j_2 + p_2 j_1)}{p_1p_2} = 
  k_2 j_1 + \frac{ k_1 j_1}{p_1} + \frac{ k_1j_2 + k_2 j_2 p_1}{p_1 p_2} = k_2 j_1 + \frac{k_1 j_1}{p_1} + \frac{j_2 k}{N}.
\]
Первое слагаемое просто будем игнорировать. Остаются ещё два. Тем самым у нас на самом деле всё должно радикально упроститься.
\[
 A_k = \frac1{p_2} \RY {j_2}0{p_2-1} \hat A(k_1,j_2) \exp \left( -2\pi i \frac{k j_2}{N} \right),\qquad
 \hat A_k(k_1,j_2) = \frac1{p_1}\RY {j_1}0{p_1-1} f_{j_2+p_2j_1} \exp\left( -2\pi i \frac{k_1 j_1}{p_1} \right).
\]

Давайте оценивать алгоритм. $\hat A$ это $O(p_1^2 p_2)$,  $A$ есть $O(p_1 p_2^2)$. Таким образом $O(p_1^2 p_2 + p_1 p_2^2)$. Если получилось так, что $p_1,p_2 \sim \sqrt{N}$, то $O(N^{\frac32})$. А если вы смогли разложить не на два, а на три множителя. Ситуацию надо доводить до абсурда, то есть когда $N=2^m$ и алгоритм оценивается, как $O(N\log\limits_2 N)$.

К этому мы вернёмся в мае, когда будем решать уравнения в частных производных.

\section{Численное дифференцирование}
Тема большая, мы её ужмём на полторы лекции. Например, ограничимся вычисление производных от функций одной переменной.

Пусть есть некоторая функция $f$, гладкая настолько, насколько нам потребуется. И есть точка $x_0$. Нам надо посчитать $k$-ю производную $f^{(k)}(x_0)$. Задача в таком виде не очень хороша, неясно, чем можно пользоваться. Будем пытаться делать как раньше. Считаем, что есть набор точек $x_1,\dots,x_n$, в которых функция известна $f(x_1),\dots,f(x_n)$.

Вариант считать производную у интерполяционного многочлена. Но суммы чудовищные у Лагранжа, у Ньютона ещё хуже. Поэтому будет чуть по-другому оформлено. Интерполяционный многочлен "--- это первый подход.

Второй подход: приблизим $f^{(k)}(x_0 \approx \RY j1n c_j f(x_j)$. От формулы потребуем, чтобы на многочленах формула была точна. На коэффициенты при этом получаются линейные соотношения. Результат получается, как бы это ни было смешно, такой же, как и при дифференцировании интерполяционного многочлена.

Давайт подставим в нашу формулу многочлен $f(x) = \RY j1m a_j x^j$.
\[
  \RY j0m a_j (x^j)^{(k)} \bigg|_{x_0} = \RY j1n c_i\bigg(\RY j1m a_j x_i^j\bigg)..
\]
Надо собрать подобные. Имеем $(x^J)^{(k)} = j(j-1)\dots(j-k+1) x^{j-k}$.

На $c_i$ система с матрицей определителя Вандермонда. Для этого, $m=n-1$. И такие коэффициенты единственны.

Примем без доказательства приятный факт, а именно, что можно каждый раз не мучиться, а говорить, что если точки будем брать определённым регулярным образом, можно пользоваться для коэффициентов пользоваться едиными формулами.

Пусть $h>0$, $h\ll1$. Набор $x_0 + i h$, $i=0,\pm 1, \pm2,\dots$ называется шаблоном. Точки расположены равномерно. В этом случае у нас будет получаться, что формула получается такой
\[
  f^{(k)}(x_0)\approx \frac{\sum\limits_i c_i f(x_i)}{h^k}.
\]
Коэффициенты для всех производных те же.

Давайте чего-нибудь посчитаем. Посчитаем $f'(x)$. Одной точки недостаточно, потому что в этом случае интерполяционный многочлен будет константой и производная будет ноль. Берём две точки, $x,x_h$ или две точки $x-h$, $x$. Тогда будем брать вместо наклона касательной, наклон секущей
\[
  f'(x)\approx \frac{f(x+h) - f(x)}h = D_+^{(1)}(h)f(x).
\]
Для другой пары точек получим другую формуру
\[
  f'(x)\approx \frac{f(x) - f(x-h)}h = D_-^{(1)}(h)f(x).
\]

Дальше нам надо увеличивать количество точек. Возьмём три точки $x-h,x,x+h$. Формально, это, конечно, парабола, потому что точек три. Можно выписать метод неопределённых коэффициентов, я выпишу результат: по факту отработают только две точки
\[
  f'(x) \approx \frac{f(x+h) - f(x-h)}{2h} = D_0^{(1)}(h) f(x).
\]

Как мы будем оценивать их эффективность? Во всех этих формулах присутствует малый параметр, если $h$ будем брать меньше, точность будет больше. Нам нужен способ оценки погрешности этих формул.

Начнём с первых формулок.
\[
  \frac{f(x+h)-f(x)}{h} = \frac1h\left( f(x) = h f'(x) + \frac {h^2}2 f''(\xi) - f(x) \right) = f'(x) + \frac h2 f''(\xi),\qquad \xi \in(x,x+h).
\]
Такая же примерно формула получится для второй формулы. Погрешность имеет линейный порядок по $h$.
\begin{multline*}
  \frac{f(x+h)-f(x-h)}{2h} = \frac1{2h} \left(  f(x) + h f'(x) + \frac{h^2}2f''(x) + \frac{h^3}6 f'''(\xi) - f(x) +h f'(x) - \frac{h^2}{2} f''(x) + \frac{h^3}{6}f'''(\xi) \right) = \\=
  f'(x) + \frac{h^2}{12} \big(f'''(\xi_1) + f'''(\xi_2)\big) = f'(x) + \frac{h^2}6 f'''(\xi_3).
\end{multline*}
Здесь $\xi_3$ получается из теоремы о среднем.

Давайте теперь поймём: есть метод построить с точностью $h$, есть метод, дающий $h^2$. Давайте попробуем общим метод выстроить. Вот считаем мы 
\[
f^{(k)}(x) = \frac{\RY j1n c_j (x_j)}{h^k} + c h^p.
\]
Мы функцию на самом деле считаем с погрешностью $f(x) \sim f^*(x)$, $|f-f^*|\sim \e$. Как будет выглядеть реальная погрешность? Формально она будет зависеть от $h$.
\[
  R(h) \le ch^p + \frac{\RY j1n|c_j|\e}{h^k} = c h^p + \frac{M\cdot \e}{h^k} = g(h).
\]
Картинка довольно грустная. Слишком маленьким $h$ мы брать не должны. Как определить оптимальную величину $h$? Определять будем, конечно, только по порядку, но результат здесь довольно поучительный.
Надо просто у $g(h)$ производную приравнять к нулю
\[
  cp h^{p-1} - \frac{ Mk\e}{h^{k+1}} = 0\qquad
  \imp\qquad
  h_0 \sim \e^{\frac1{p+k}},\quad g(h_0)\sim \e \e^{\frac p{p+k}}.
\]
Нам $k$ дано свыше, это заказ. Если мы $p$ выберем $1$ для $k=1$, то $h_0 = \e^{\frac12}$ и $g(h_0)\sim \e^{\frac12}$. Если же для $k=1$, $p=2$, то $h_0\sim \e^{1/3}$ и $g(h_0) \sim \e^{\frac23}$. И чем выше порядок точности формулы, тем точнее мы сможем посчитать производную.

Теперь давайте построим формулу для порядков производных повыше. Возьмём $k=2$. Тут минимум три точки, и мы получаем некоторый выигрыш, когда точки расположены симметрично. Давайте возьмём $x-h$, $x$, $x_h$. Давайте возьём 
\[
  D^{(1)}_-(h) D_+^{(1)}(h)f(x) = D_-^{(1)}(h)\left( \frac{f(x+h)-f(x)}{h} \right) = \frac{f(x-h) - 2f(x) + f(x+h)}{h^2}\equiv D^{(2)}(h) f(x).
\]
Эту формулу надо знать в лицо.
\begin{multline*}
  D^{(2)}(h)f(x) = \frac1{h^2}\left(  f - h f' + \frac{h^2}2 f'' - \frac{h^3}6 f''' +\frac{h^4}{24}f^{(4)}(\xi_1) - 2f + f + hf' + \frac{h^2}2 +\frac{h^3}6 f''' + \frac{h^4}{24}f^{(4)}(\xi_2) \right) = \\
 = f''(x) + \frac{h^2}{12}f^{(4)}(\xi).
\end{multline*}
Формула получилась лучше, чем мы ожидали (за счёт симметрии): она верна для многочленов до третей степени. Других я писать и не буду, поэтому обозначение $D$ без всяких плюсов-минусов.

Дальше можно двигаться в нескольких направлениях: добавлять количество узлов и уточнять формулы или учиться считать третью производную.
\[
  f'''(x)\approx  D_0^{(1)}(h) D^{(2)}(h)f(x)
\]
Здесь получатся узлы $x-2h,x-h,x+h,x+2h$. Обоснуйте дома разложением в ряд Тейлора, насколько формула точна.

Теперь к другому вопросу. Вот пусть мы $k$ зафиксировали. Как безболезненно увеличивать число узлов. Мы сейчас впервые встретимся таким подходом, который называется «правило Рунге оценки погрешности». Предположим мы с вами затеяли вычислять $k$-ю производную
\[
  f^{(k)}(x) = D^{(p)}(h)f(x) + ch^p + O(h^{p+1}).
\]
Здесь мы вдруг добавили не остаточный член в промежуточной точке, а $O$. В качестве упражнения убеиться, что эта формула уточняется до $O(h^{p+2})$ в наших формулах.

Нам надо решить задачу с точностью $\e$. Надо подбирать $h^p\sim \e$. Чем это плохо: $c$ неизвестно, вдруг оно здоровенное и этого мало. Или вдруг $c$ мало и вы выбрали сильно маленький шаг, а это плохо, мы уже выяснили, сильно малый шаг брать нельзя из-за машинной точности.

Вот что предлагает Рунге. Пусть $0<q<1$. Тогда
\[
  f^{(k)}(x) = D^{(p)}(qh)f(x) + c (qh)^p + O(h^{p+1}).
\]
Идея такова, что $c$ будет либо такая же, либо отличаться будет так, что разницу можно загнать в $O(h^{p+1})$. Вычтем одно из другого
\[
  ch^p = \frac{ D^{(p)}(qh)f(x) - F^{(p)}(h) f(x)}{1- q^p} + O(h^{p+1}) = \rho(h) + O(h^{p+1}).
\]
Давайте на примере $q = \frac12.$ Тогда считаем для $h$ и для $h/2$. Получаем $\rho(h)$. Если плохая погрешность, давайте считать $\rho(h/2)$. И так делим, пока погрешность нас не устроит.

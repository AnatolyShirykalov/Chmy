\section{Линейные уравнения в частных производных}
Мы сформулировали аппроксимацию, устойчивость и следующую из них сходимость (это следование мы доказали).

Общий вид задачи $Lu = f, lu = \varphi$. Всё происходит в области $\Omega$.

Разностная задача на сетке
\[
\begin{cases}
L_h\omega_h = f_h, & \Omega_h;\\
l_h\omega_h = \phi_h.
\end{cases}
\]

Нам надо разобраться со сходимостью. Мы под ней понимали, что $\big\|[u]_h - u_h\big\|_{\Omega_h}\to 0$. Теперь у нас теорема
\begin{The}
 Из аппроксимации и устойчивости следует сходимость.
\end{The}
Работает всегда, но докажем только для линейных.
\begin{Proof} Пусть у нас аппроксимация с порядком $p = (p_1,p_2)$ (это мультииндекс).
  Проделаем такой фокус. Сеточный оператор применим вот к такой штуке
\[
  L_h\big(\underbrace{u_h-[u]_h}_{w_h}\big) = L_hu_h - L_h[u]_h\pm [Lu]_h = 
  \big([Lu]_h - L_h[u]_h\big) + \big(f_h - [f]_h\big) = z_h.
\]
Аналогично для оператора на границе
\[
  l_h\big(u_h - [u]_h\big) = \big([lu]_h - l_h[u]_h\big) + \big(\phi_h - [\phi]_h\big) = \zeta_h.
\]

У нас получилось
\[
\begin{cases}
L_hw_h = z_h,&\Omega_h;\\
l_hw_h = \zeta_n.
\end{cases}
\]

Для этой задачи у нас есть оценка нормы решения через нормы правых частей.

Так как у нас есть устойчивость, то $\|w_h\|\le C\big(\|z_h\| + \|\zeta_h\|\big)$. А так как есть аппроксимация, то это всё $\le c_1 h^p$.
\end{Proof}

Обычно с аппроксимацией у нас будет всё в порядке. А вот наличие устойчивости будет представлять интерес.

\subsection{Задачи}
Рассмотрим первую простейшую задачу
\[\begin{cases}
  \CP ut - \CP ux = f(x,t);\\
  u(x,0) = \phi(x).
\end{cases}
\]
Решать будем в полосе $[0,T]$. Вы должны быть готовы на экзамене ответить на вопрос, что будет, если перед $\CP ux$ добавить множитель $a$.

Выбираем шаг $h$ и шаг $\tau$ так, что по времени отрезок разбивается на целое число частей. Будем приближать $u(mh,n\tau) \sim u_m^n$.

Здесь есть два простейших способа замена дифференциального оператора на разностный. Разности вперёд, разность назад.
\begin{gather}
\begin{cases}
\frac{u_m^{n+1}}{\tau} - \frac{u_{m+1}^n}{u_m^n} = f_m^n;\\
u_m^0 = \phi_m.
\end{cases}\\
\begin{cases}
\frac{u_m^{n+1}}{\tau} - \frac{u_{m}^n}{u_{m-1}^n} = f_m^n;\\
u_m^0 = \phi_m.
\end{cases}
\end{gather}

В дифференциальной задаче на линиях $x+t=\const$ значения постоянны. Значение на точке $x=0,t=T$ определяется значением в точке $t=0,x=T$. Если мы в последней внесём возмущение, оно перейдём в точку $x=0,t=T$.

В приближении второго типа значение в рассматриваемой точке определеяется значением в трегольнике слева. Возмущение в точке $t=0,x=T$ не передаётся в точку $x=0,t=T$. Отбрасываем вторую схему. 

Рассмотрим первую схему. Если $\frac\tau h\le 1$, то задача по крайней мере не разваливается (характеристика попадает в треугольник, от которого зависит точка последнего слоя).

Итак у нас есть подозрение, что первая схема годится. Но мы ещё ничего не доказали. Введём
\[
  \|u_h\| = \max\limits_n\sup\limits_m|u_m^n|,\quad 0\le h\le N, N\tau = T.
\]
Такая же для $\|f_h\| = \max\limits_n\sup\limits_m|f_m^n|$. И ещё
\[
  \|\phi_h\| = \sup\limits_m|\phi_m|,\quad \|u^n\| = \sup\limits_m = |u_m^n|.
\]

Если нам нужно считать не на всей числовой прямой по $x$ верхний слой, то нижний слой задаём на большем промежутке, чем верхний.

\begin{The}
Если $\tau/h\le 1$, то схема один устойчива.
\end{The}
\begin{Proof}
 Пусть $\tau/h = q\le 1$. Тогда $u_m^{n+1} = (1-q) u_m^n  + q y_{m+1}^n + \tau f_m^n$.
\[
  |u_m^{n+1}|\le (1-q)|u_m^n| + q |u_{m+1}^n| + \tau |f_m^n| \le \|u^n\| (1 - q + q) +\tau \|f_h\|.
\]
Надо перейти к супремуму
\[
  \|u^{n+1}\|\le \|u^n\| + \tau\|f_h\|,\quad 
  \|u^{n}\|\le \|u^{n-1}\| + \tau\|f_h\|,\quad\dots\quad  
  \|u^{1}\|\le \|u^0\| + \tau\|f_h\|.
\]
Мы это дело всё сложим
\[
  \|u^{n+1}\|\le \|\phi_h\| + (n+1)\tau\|f_h\| \le \|\phi_h\| + T \|f_h\|.
\]
 Итак, получили
\[
  \|u_h\|\le C\big(\|\phi_h\| + \|f_h\|\big),\quad C = \begin{cases}
1,& T<1;\\ T, T\ge 1.
\end{cases}
\]
Интересно посмотреть, что будет, если добавить в уравнение параметр $a$.
\end{Proof}

Смогли отбросить одну из схем, потому что знали, как решать уравнение. Но не всегда же мы умеем уравнение решать.

\subsection{Спектральная устойчивость}
Вернёмся. Распишем нашу задачу
\[
\begin{cases}
  \CP ut - a\CP ux = f;\\
   u(x,0) = \phi.
\end{cases}
\]
Будем считать, что у нас $a>0$. Напишем аналог первой схемы.
\[
\begin{cases}
  \frac{u_m^{n+1} - u_m^n}{\tau} - a\frac{u_{m+1}^n - u_m^n}{h} = f_m^n;\\
  u_m^0 = \phi_m.
\end{cases}
\]

На этом примере я сейчас расскажу спектральный признак. Но сначала рассмотрим такую задачу
\[
\begin{cases}
  \frac{u_m^{n+1} - u_m^n}{\tau} - a\frac{u_{m+1}^n - u_m^n}{h} = 0;\\
  u_m^0 = e^{im\alpha}.
\end{cases}
\]
Здесь $i^2 = -1$, $\alpha\in[0,2\pi]$. Будем говорить, что имеется выполнение спектрального признака, если решение будет устойчивость (ограничено по норме сверху через правую часть и правую часть на границе) любой такой задачи. То есть устойчивость в такой задаче это просто ограниченность решения.
\[
  u^1_m = u_m^0 + \frac{a\tau}{h}(u_{m+1}^0 - u_m^0) = e^{im\alpha} + \frac{a\tau}h\big(e^{i(m+1)\alpha} - e^{im\alpha}\big) = e^{im\alpha}\left( 1 - \frac{a\tau}h + \frac{a\tau}he^{i\alpha} \right) = e^{im\alpha}\lambda.
\]
Аналогично можно показать, что $u_m^2 = u_m^1 \alpha = e^{im\alpha}\lambda^2$. И в конце концов $u_m^n= \lambda^n e^{im\alpha}$. Получили, что у нашей задачи $\lambda$ выступает в роли собственного значения. Отсюда название признака «спектральный».

Итак, если $|\lambda|\le 1$, то спектральный признак выполнен. Это сильная форма.

Но если мы работаем в полосе, то можно задать лишь слабую форму $|\lambda|\le 1+ C\tau$. В этом случае
\[
  |u_m^n| (1 + c\tau)^n\le \left( 1 + c\frac\tau N \right)^N\le e^{cT}.
\]

На слабую форму у нас времени особо нет. Будем говорить о сильной.
\subsection{Будет ли спектральная устойчивость в нашей задаче}
Итак
\[
  \lambda 1 - \frac{a\tau}h + \frac{a\tau}h e^{i\alpha}.
\]
Это окружность с центром $1 - \frac{a\tau}h$ и радиусом $\frac{a\tau}h$. Нам нужно, чтобы вся эта окружность лежала внутри единичного круга.

Если $\frac{a\tau}h \le 1$, то всё хорошо.

\section{Уравнение теплопроводности}
Уравнение теплопроводности имеет вид
\[
  \CP ut = a^2 \CP{^u}{x^2} + f(x,t).
\]
Если мы рассматриваем задачу в полосе $[0,T]$, то достаточно взять начальное условие $u(x,0) = \phi(x)$.

Будем брать следующую схему
\[
 \begin{cases}
  \frac{ u_m^{n+1} - u_m^n}{\tau} = a^2 \frac{ u_{m+1}^n - 2u_m^n + u_{m-1}^n}{h} + h_m^n;\\
  u_m^0 = \phi_m.
\end{cases}
\]
На ряду с этой, рассмотрим схему
\[
 \begin{cases}
  \frac{ u_m^{n+1} - u_m^n}{\tau} = a^2 \frac{ u_{m+1}^{n+1} - 2u_m^{n+1} + u_{m-1}^{n+1}}{h} + h_m^{n+1};\\
  u_m^0 = \phi_m.
\end{cases}
\]
В первой схеме всё логично, чтобы получить информацию о верхнеё точке, пользуемся информацией уже известных нижних.
Вторая схема какая-то странная.

Будем брать $u_m^0 = e^{im\alpha}$ будем получать $u_m^n = \lambda^n e^{im\alpha}$. Для первой схемы выполнение спектрального признака только при $\tau\sim h^2$ (задача для читателя получить точную оценку). Во второй схеме получим безусловную спектральную устойчивость.


А если мы будем рассматривать задачу на прямоугольнике $[0,X]\times [0,T]$. Нужны ещё условия на левой и правой границе $\mu_1(t)$ и $\mu_2(t)$. В этом случае какие условия у нас добавятся.

Условие здесь будет общее для обеих схем
\[
  u_m^0 = \phi_m, u_0^n = \mu_1^n,\ u_M^n = \mu_2^n.
\]

В первой схеме у нас $O(M)$ арифметических операций для расчёта каждого слоя. На второй схеме надо решать систему уравнений. $u^{n+1}= (u_1^{n+1},\dots,u_{M-1}^{n+1})$. Хочу написать $Au^{n+1} = F^n$. 
\[
 - \frac{a^2\tau}{h^2} u_{m-1}^{n+1} + \left(1 + \frac{2 a^2\tau}{h^2}\right) u_m^{n+1} - \frac{a^2\tau}{h^2} u_{m+1}^{n+1} = \tau f_m^{n+1} + u_m^n.
\]

Как будет выглядеть матрица. Она будет трёхдиагональная. Решаем методом прогонки, получаем $8M + O(1)$. То есть она по порядку аримфетических операций не уступает явному методу.

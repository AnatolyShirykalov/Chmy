Мы на чём остановились. Рассматривали уравнение теплопроводности
\[
  \CP ut = \CP{^2u}{x^2} + f.
\]
В прямоугольной области с соответствующими краевыми условиями.
\[
  \frac{u_m^{n+1}- u_m^n}{\tau} = \sigma \Lambda u_m^{n+1} + (1-\sigma) \Lambda u_m^n + f_m^n,\quad
  \sigma\in[0,1].
\]

Если рассматриваем аппроксимацию в точке $(m,n+0.5)$, И при этом берём $f_m^n = f\big(mh,(n+0.5)\tau\big)$ и $\sigma=0.5$. В этом случае $O(\tau^2+h^2)$.

Возникает проблема. Устойчивость и аппроксимацию будут записаны в разных нормах, то есть по разным точкам. Со сходимостью видимо тоже будет какая-то беда.

У нас будет упрощённая ситуация. Возьмём упрощённую задачу. Мы возьмём правую часть нулём. И нашу область вместе со своей сеточкой. Будем говорить, что у нас $u_0^n=0=u_M^n=0$, $Mh=X$. Единственное, что у нас будет, это начальное условие $u^0_m=\phi_m$.

У нас было понятие нормы на слое. Но у нас был максимум, а теперь возьмём
\[
  \|u^n\| = \sqrt{\RY m1{M-1}(u_m^n)^2h}
\]
Это аналог нормы в $L_2$. Эта норма не имеет отношение к той норме, которая была в аппроксимации. 
\begin{Def}
Устойчивость в этой норме
\[
  \max\limits_{0\le n\le N} \|u^n\|\le c\|u^0\|
\]
\end{Def}

Давайте исследовать. Норма не совсем обычная. Те приёмы, которые у нас были, здесь не пройдут. Но вот этот оператор $\Lambda$, который разностная вторая производная, мы его в прошлый раз изучили. Мы выписали его собственные значения и собственные функции.

У нас $u^n = (0,u_1^n,\dots,u_{M-1}^n)$. Пусть $E$ "--- единичный оператор
\begin{equation}\label{eqeq}
  (E - \sigma\tau \Lambda)u^{n+1} = \big(E + (1-\sigma)\tau\Lambda\big)u^n.
\end{equation}
Собственные значения оператора $\Lambda$ назовём $-\alpha_k$, где $\alpha_k = \frac4{h^2}\sin^2\frac{\pi k h}2$. А собственные вектора $v^{(k)}\colon c^{(k)}_n = \sin\frac{\pi k n h}{X}$.

Давайте решение будем раскладывать по этим собственным векторам. В нашем равенстве \eqref{eqeq} собственные векторы правой и левой часть такие же, как и у оператора $\Lambda$, а собственные значения чуть другие. Можем написать
\[
  u^{n+1} = \underbrace{(E-\sigma\tau \Lambda)^{-1}\big(1+(1-\sigma)\tau\Lambda\big)}_S u^n.
\]
Буквой $S$ обозначили оператор перехода к новому слою. Собственные значения его мы сейчас найдём. Раскладываем начальное условие по собственным векторам
\[
  u_m^0 = \RY k1{M-1} c_k \sin\frac{\pi kmh}X,\quad m=1,\dots,M-1.
\]
На первом слое
\[
  u_m^1 = S u_m^0 = \RY i1{M-1} \lambda_k c_k\sin\frac{\pi k m h}X.
\]
Отсюда собственные значения
\[
  \lambda_k = \frac{1-(1-\sigma)\tau\alpha_k}{1+\sigma\tau\alpha_k}.
\]
Что у нас будет на втором слое, всё будет по аналогии
\[
  u_m^n = \RY k1{M-1} \lambda_k^n c_k \sin().
\]
Ага, значит, нам достаточно, чтобы $|\lambda_k|<1$. Значит, $\alpha_k\in\left(0,\frac4{h^2}\right)$. Отсюда
\[
  -1\le \frac{1-(1-\sigma)\tau\alpha_k}{1 + \sigma\tau\alpha_k}.
\]
Перепишем
\[
  -1 - \sigma\tau\alpha_k\le 1- (1-\sigma)\tau\alpha_k\le 1+\sigma\tau\alpha_k.
\]
Правое равенство очевидно. Осталось только левое
\[
  \tau(1-2\sigma)\alpha_k\le 2.
\]
Если $\sigma=0.5$, то всё заведомо выполнено. Программу минимум сделали. Но мы сделаем подробнее
\begin{itemize}
\item Случай $\sigma\in[0.5,1]$. Тогда устойчивость для всех $h,\tau$.
\item Если $\sigma\in[0,0.5)$. Тогда нужно
\[
  \tau\le \frac2{(1-2\sigma)\max\alpha_k}.
\]
Потребуем, чтобы $\tau < \frac{h^2}{2(1-2\sigma)}$.
\end{itemize}

\subsection{Сходимость}
Сходимость будет по совсем хитрой норме. Я этот факт доказывать не буду. Доказательство требует совсем другой техники, это было бы уже слишком. Норма тут такая
\[
  \|v\|_1 = \sqrt{\RY m0{M-1}h\left( \frac{v_{m+1} - v_m}h \right)^2}.
\]
Здесь $v = (0,v_1,\dots,v_{M-1},0)$.
\[
  \|u_h\|_1 = \max\limits_n\|u^n\|_1.
\]
В такой норме сходимость будет с порядком $O(\tau^2+h^2)$.

\section{Стационарные задачи}
Времени никакого нет, все переменные равноправны. Основная задача, которую мы будем рассматривать. А может на ней-то мы и остановимся.

Пусть у нас будет прямоугольник в плоскости $(x_1,x_2)$ с углом в нуле, ширины $l_1$, высоты $l_2$. Уравнение
\[
  -\Delta u = f,\quad x = (x_1,x_2)\in\Omega\quad u\big|_{\dl\Omega} = \alpha.
\]

Пусть $h_1N1 = l_1$, $h_2N_2 = l_2$.

Рассмотрим операторы
\[
  \Lambda_1 u_{m,n} = \frac{u_{m-1,n} - 2u_{m,n} + u_{m+1,n}}{h_1^2},\quad
  \Lambda_2 u_{m,n} = \frac{u_{m,n-1} - 2u_{m,n} + u_{m,n+1}}{h_2^2}.
\]

И рассмотрим схему
\[
  -(\Lambda_1+\Lambda_2)u_m^n = f_m^n.
\]
При этом $u_{m,0}=\alpha_{m,0}$, $u_{m,N2} = \alpha_{m,N_2}$. И далее

Схема такова, что углы нигде не используются. Для эллиптических задач извесная проблема таких особенностей.

Что требовала у нас устойчивость: она требовала существования и единственности решения. Ну ладно. Как нам хранить в памяти точки? Нужно их как-то перенумеровать, чтобы был линейный порядок.

Давайте считать, что у нас $l_1=l_2$, $h_1=h_2=h$, $N_1=N_2=N$. Как выглядит наша схема в этом случае. Очень симпатично выглядит
\[
  \frac{ - u_{m-1,n} - u_{m+1,n} + 4u_{m,n} - u_{m,n-1} - u_{m,n+1}}{h^2} = f_{m,n}.
\]

Давайте нумеровать слева-направо и снизу вверх. Можно и в других направлениях, но лучше не получится. Наша нумерация окажется оптимальной. Первый слой нумеруем $1,2,\dots,N-1$, далее $N,N+1,\dots,2N-2$. Я хочу записать нашу схему в виде
\[
  A \ve u = \ve F.
\]
Как выглядит наша матрица
\[
  A = \frac1{h^2}\begin{pmatrix}
  4 & -1 & 0 &\dots &0 &-1\\
  -1 & 4 & -1 & 0 & \dots & 0 & -1\\
\end{pmatrix}<++>
\]

Получится ленточная матрица размерности $A = \big((N-1)^2\times (N-1)^2\big)$. Мы покажем, что эта матрица положительно определена. Мы запустим метод Гаусса сложностью $O(N^4)$. А затем если $N=2^p$ сделаем метод сложности $O(N^2\log N)$.

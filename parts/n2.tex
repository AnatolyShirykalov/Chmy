
Давайте теперь поговорим. Решали нелинейные скалярные уравнения. Теперь нас интересует ситуация, когда вообще говоря
\[
  \begin{cases}
f_1(x_1,\ldots,x_n) = 0;\\
\dotfill\\
f_m(x_1,\dots,x_n)=0.
\end{cases}
\]

Мы должны с грустью убедиться, что почти все наши методы не переносятся. Даже метод половинного деления.

Было у нас $f(x)=0$, было решение $\ol x\colon f(\ol x)=0$. Пусть нам известно $x_n$. Как получить $x_{n+1}$? Разложим в ряд
\[
  f(x)\approx f(x_n) + (x-x_n) f'(x_n) = 0.
\]
Положим это принудительно равным нулю. А тот $x$, что оттуда получится и назовём через $x_{n+1}$. Это метод Ньютона или метод касательных
\[
  x_{n+1} = x_n - \frac{f(x_n)}{f'(x_n)}.
\]

Как нам это обобщить? Пусть есть $X$ и $Y$ "--- линейные нормированные пространства. Пусть есть нелинейный оператор $F(x)\colon X\to Y$. Пусть в пространствах есть нормы $\|\cdot\|_X$, $\|\cdot\|_Y$. Введём одно из возможных определений производной для этого самого оператора.
\begin{Def}
Линейный оператор $P\colon X\to Y$ мы назовём производной, то есть $P = F'(x)$, если
\[
  \big\|F(x+h) - F(x) - Ph\big\|_Y = o\big(\|h\|_X\big).
\]
при $\|h\|_X\to 0$. Это одно из возможных определений. Оно нам будет удобно.

В нашем случае $X$ и $Y$ это $\R^m$. Здесь $x = (x_1,\dots,x_m)^T\in \R^m$. Что такое оператор $F$, это
\[
  F(x) = \big(f_1(x),\dots, f_m(x)\big)^T.
\]
Что такое производная $P$? Это есть матрица
\[
  P = F'(x) = \left[ \CP{f_i}{x_j} \right]^m_{i,j=1}.
\]
\end{Def}

Мы решаем уравнение $F(x)=0$, где $0\in Y$. Пусть $\ol x $ "--- решение. Пусть $x^n$ "--- какое-то приближение. Пусть существует $F'(x_n)$. Что тогда у нас с вами получится. Будем считать, что $x_n$ достаточно близок к корню, можем перефразировать (заменить о определении производнойй $x$ на $x^n$, а $x+h$ на $\ol x$)
\[
  \big\|\underbrace{F(\ol x)}_0 - F(x^n) - F'(x^n)(\ol x - x_n)\big\|_Y = o\big(\|\ol x - x^n\|_X\big).
\]
Итого у нас
\[
  -F(x^n) - F'(x^n)(\ol x-x^n)\approx 0.
\]

Если мы захотим поставить там прямо равенство, придётся отказаться от $\ol x$. Поставим вместо $\ol x$ следующее приближение
\[
  F(x^n) = F'(x^n)(x^{n+1} - x^n) = 0.
\]

Как выудить отсюда $x^{n+1}$? Надо домножить на обратную матрицу (предполагаем, что она есть). Выйдет у нас окончательно базовое соотношение
\[
  x^{n+1} = x^n - \big[F'(x^n)\big]^{-1}F(x^n).
\]
Это тоже принято называть методом Ньютона. Размерность здесь не важна.

А теперь перед нами вопросы. Это вообще осуществимо? Сходится ли и с какой скоростью?

Редкая ситуация, когда формулировка теоремы зайём больше места, чем доказательство.

Обозначим $\Omega_a = \big\{x\colon x\in X,\ \|x-\ol x\|_X<a\big\}$.
\begin{The}
Пусть существуют такие постоянные $a,a_1,a_2\colon a>0, 0\le a_1,a_2<\infty$ и существует $F'(x)$ для всех $x\in \Omega_a$, для которых
\begin{roItems}
\item $\forall\ x\in \Omega_a\pau \Big\|\big(F'(x)\big)^{-1}\Big\|_X \le a1$ (исключает много всего, в том числе кратные корни; $x^2=0$ нехорошо);\\
\item $\forall\ u_1,u_2\in \Omega_a\pau 
\big\|F(u_1) - F(u_2) - F'(u_2)(u_1-u_2) \big\|_Y \le a_2\|u_2 - u_1\|_X^2$ (не просто о-малое, а убывает как квадрат, то $x^{\frac32}=0$ тоже не хорошо).
\end{roItems}
Если всё это хозяйство выполнено, то обозначим $c = a_1a_2, b = \min\{a,c^{-1}\}$. Тогда отсюда следует, что если $x^0\in \Omega_b$, то метод Ньютона сходится и выполнено 
\[
  \|x^n-\ol x\| \le c^{-1}\big(c\|x^0-\ol x\|_X\big)^{2^n},
\]
то есть $\|x^{n+1} - \ol x\| \le \const \|x^n-\ol x\|^2$.
\end{The}
На пальцах условия: пусть нет кратных корней и всё достаточно гладкое.
\begin{Proof}
 Считаем, что у нас $x^0\in\Omega_b$. Это похоже на то, что мы делали с методом прогонки. Попытаемся показать, что из того, что $x^n\in \Omega_b$, следует, что и $x^{n+1}\in \Omega_b$. А в итоге докажем попутно сразу всё.

Положим $u_1 = \ol x$, $u_2 = x^n$. (Формулой Ньютона пока не пользуюсь.) Тогда второе условие даёт нам
\[
  \big\|\underbrace{F(\ol x)}_0 - F(x^n) - F'(x^n)(\ol x - x^n)\big\|_Y\le a_2\|x^n-\ol x\|^2.
\]
А дальше в скобках что стоит? Вспоминаем метод Ньютона
\[
  F(x^n) + F'(x^n)(x^{n+1} - x^n) = 0.
\]
Из этой расчётной формулы выудим $F(x^n) = -F'(x^n)(x^{n+1} - x^n)$.  И подставим
\[
  \big\|F'(x^n) (x^{n+1} - x^n) - F'(x^n)(\ol x - x^n)\big\|\le a_2\|x^n - \ol x\|^2_X.
\]
И мы уже получили то, с чем можно работать
\[
  \big\|F'(x^n)(x^{n+1} - \ol x)\big\|_Y\le a_2 \|x^n - \ol x\|_X.
\]
Если бы не было $F'(x^n)$, это была бы написана квадратичная сходимость. Покажем, что $x^{n+1}\in\Omega_b$.
\begin{multline*}
  \|x^{n+1} - \ol x\|_X = \Big\|\big(F'(x^n)\big)^{-1} F'(x^n)(x^{n+1} - \ol x\Big\|_X\le \\\le
  \Big\|\big(F'(x^n)\big)^{-1}\Big\| \cdot \big\|F'(x^n)(x^{n+1}-\ol x)\big\|\le a_1a_2 \|x^n-\ol x\|^2 = c\|x^n - \ol x\|^2 < cb^2 = (cb) b< b.
\end{multline*}

Надо было остановиться на моменте $\|x^{n+1} - \ol x\|_X \le c\|x^n - \ol x\|^2_X$.

Обозначим $q_n = c\|x^n - \ol x\|$. У нас получилось, что $\|x^{n+1} - \ol x\|_X \le c\|x^n - \ol x\|^2_X$. Давайте домножим наше соотношение на $c$. Тогда
\[
  q_{n+1} \le q_n^2.
\]

Теперь попытаемся показать, что раз такое условие выполнено, то $q_n\to 0$. Но ведь $q_0 = c\|x^0 - \ol x\|<cb <1$.

Я утверждаю, что $q_n \le (q_0)^{2^n}$. Это совпадает с утверждением теоремы. Доказывать будем по индукции. База $n=0$. Пусть $q_k\le (q_0)^{2^k}$. Тогда
\[
  q_{k+1}\le q_k^2 \le \big((q_0)^{2^k}\big)^2 = q_0^{2^k}.
\]
\end{Proof}

К сожалению, даже для скалярного уравнения, трудно поймать начальное приближение, для которого сходимость будет именно квадратичная. Часто начальное приближение находят опытным путём, как раз проверяя условия теоремы.

\section{Дифференциальные уравнения}
Рассмотрим для начала скалярную задачу Коши
\[
  \begin{cases}
  y'(x) = f\big(x,y(x)\big);\\
  y(x_0) = y_0.
\end{cases}
\quad
 x\in[x_0,X].
\]
Часто нас будет интересовать поведение решения не везде, в только лишь в конечной точке.

Здесь есть несколько подходов. Мы будем искать не функцию, которая хорошо приближает решение. А будем искать, как решение ведём себя в конкретно заданных точках.

Возьмём на отрезке сетку шагом $h$, $x_n = x_0 + hn$, $x_0 + hN = X$. Если значения функции на сетке мы знаем, мы можем заменить производные разностными соотношениями. Скажем, что
\[
  y'(x_n) = \frac{y(x_{n+1}) - y(x_n)}{h} + O(h).
\]

После этого можем записать $y(x_{n+1}) = y(x_n) + h f\big(x_n,y(x_n)\big) + O(h^2)$. К этому присовокупить $y(x_0) = y_0$.

Если бы мы знали $O(h^2$, то мы бы просто могли посчитать $y(x_{n+1})$. У нас возникает соблазн $O(h^2)$ просто убрать. Тогда мы не будем иметь право результать обозначать $y(x_n)$. Будем писать $y_n$. Получаем приближённый метод решения. Это так называемый метод Эйлера
\[
  y_{n+1} = y_n + h f(x_n,y_n),\quad y_0.
\]
Вопрос, насколько метод эффективен. Что у нас будет происходить на самом деле? Первая ошибка будет порядка $h^2$. если бы второй шаг начинали с правильной точки, то тоже был бы $h^2$. Но мы начали второй шаг с неправильной точки. Насколько же быстро ошибка будет разрастаться?

Но метод реализуем хотя бы. Здесь нет деления на ноль, число шагов конечно, в бесконечность не уйдём. Вопрос только в том, что мы здесь получим в результате.

Нам нужно ввести некоторые определения, а именно классификацию погрешностей.

Введём глобальную погрешность $E_n = y_n - y(x_n)$. И введём понятие локальной погрешности $e_n = y_n - \Til y(x_n)$, где $\Til y$ определяется, как решение задачи Коши
\[
  \Til y\colon \begin{cases}
  \Til y'(x) = f\big(x,\Til y(x)\big);\\
  \Til y(x_{n-1}) = y_{n-1}.
\end{cases}
\]

Теперь теоремка. Я упрощаю, а можно было и более жёстко дать.

\begin{The}
  Пусть $f(x,y)\in C^2$. Тогда отсюда следует, что $e_n = O(h^2$.
\end{The}
\begin{Proof}
  Рассмотрим вот такую задачу
\[
  \begin{cases}
 y' = f(x,y);\\
 y(x_n) = y_n.
\end{cases}
\]
При этом $y_{n+1} = y_n + h f(x_n,y_n)$. Что есть $y(x_{n+1}) = y(x_n+h) = y(x_n) + h \underbrace{y'(x_n)}_{f\big(x_n,y(x_n)\big)} + \frac{h^2}2y''(\xi)$.

Отсюда у нас
\[
  e_{n+1} = y_{n+1} - y(x_{n+1}) = -\frac{h^2}2y''(\xi).
\]
Собственно, это даже не доказательство, а так просто.
\end{Proof}

Теперь исследуем глобальную погрешность. Тут начальные требования будут Жёстче.
\begin{The}
Пусть $\forall\ x\in [x_0,X]\pau |y''|\le M$, $\big|f(x,\ol y) - f(x,\ol{\ol y})\big| \le L |\ol y - \ol{\ol y}|$ Тогда
\[
  E_n = O(h),\quad n=1,\dots, N.
\]
\end{The}
\begin{Proof}
Давайте попробуем это доказать.
Мы можем расписать то, что у нас уже было
\[
  y(x_{n+1}) = y(x_n) + h y'(x_n) + \frac{h^2}2y''(\xi_n) = 
  y(x_n) + h f\big(x_n,y(x_n)\big) + \frac{h^2}2y''(\xi_n).
\]
Приближённое у нас есть $y_{n+1} = y_n + h f(x_n,y_n)$. И давайте вычтем
\[
  E_{n+1} = -\big(y(x_{n+1}) - y_{n+1}\big) = -\bigg(-E_n + h \Big(f\big(x_n,y(x_n)\big) - f(x_n,y_n)\Big) + \frac{h^2}2y''(\xi_n)\bigg).
\]
Таким образом, 
\[
  |E_{n+1}|\le |E_n| + h L |E_n| + \frac {h^2}2 M.
\]

Обозначим, $A= 1+ Lh$, $B = \frac{h^2}2$. Тогда $|E_{n+1}| = A E_n + B$. И отсюда
\[
  |E_n| \le A |E_{n-1}| + B \le A^n E_0 + \bigg(\RY k1{n-1} A^k\bigg)B.
\]
При этом $E_0 = y_0 - y(x_0)=0$. Рассмотрим $A=1$  в качестве задачи. А если $A\ne 1$, то 
\[
  |E_n| \le \frac{A^n-1}{A-1} B.
\]

Мы знаем, что $1+x\le e^x$. При этом $A^n = (1+hL)^n\le e^{Lhn} = e^{L(x_n-x_0)}$. Тогда
\[
  |E_n| \le \left( \frac{e^{L(x_n-x_0)} - 1}{1+ Lh - 1} \right)\frac{Mh^2}2 \le \frac{h M}{2L} \big(e^{L(X-x_0)}-1\big) = O(h).
\]
\end{Proof}<++>

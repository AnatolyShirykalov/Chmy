Теперь вместо $a$ поставлю единичку. Задача в конечной области $[0,T]\times[0,X]$.
\[
  \CP ut = \CP {^2u}{x^2} + f.
\]
Есть граничные условия $u(x,0) = \phi(x)$, $u(X,t) = \mu_2(t)$, $u(0,t) = \mu_1(t)$. Шаг $M h = X$, $N\tau = T$.

Выписали две схемы
\[
  \frac{ u_m^{n+1} - u_m^n}{\tau} = \frac{ u_{m-1}^n - 2 u_m^n  u_{m+1}^n}{h^2} + f_m^n;\qquad
  u_m^0 = \phi_m,\quad u_0^n = \mu_1^n,\quad u_M^n = \mu_2^n.
\]
Вывод, который мы сделали. Если условие $\tau\le \frac{h^2}2$ не выполнено, то схема неустойчива.

Теперь сформулируем теорему.
\begin{The}
Если $\tau\le \frac{h^2}2$, то схема устойчива.
\end{The}
Соглашение о нормах прежние, то есть $\|u_h\| = \max\limits_{m,n}|u_m^n|$, $\|u^n\| = \max\limits_m|u_m^n|$, $\|\phi_h\| = \max\limits_m|\phi_n|$, $\|\mu_k\| = \max\limits_n|\mu_k^n|$.
\begin{Proof}
Нужно показать, что норма разности решений разностной задачи оценивается через норму разности начальных условий. Обозначим $\frac\tau{h^2} = \rho$. Тогда
\[
  u_m^{n+1} = (1-2\rho)u_m^n + \rho u_{m-1}^n + \rho u_{m+1}^n + \tau f_m^n.
\]
Предполагаем, что $\|u^{n+1}\| = |u_0^{n+1}|\bigvee |u_M^{n+1}|$. В этом случае $\|u^{n+1}|\le \max(\|\mu_1\|,\|\mu_2\|)$.

Теперь пусть максимум достигается внутри $\|u^{n+1}\| = |u_{m_0}^n|$, где $1\le m_0\le M-1$.
\[
  \|u^{n+1}\|\le\max\limits_m\Big(\big|(1-2\rho)u_m^n + \rho u_{m-1}^n + \rho u_{m+1}^n + \tau f_m^n\big|\Big) \le (1-2\rho + 2\rho)\|u^n\| + \tau \|f_m^n\|.
\]
Таким образом, $\|u^{n+1}\| \le \max\big(\|\mu_1\|,\|\mu_2\|,\|u^n\| + \tau\|f_h\|\big)$.

Дальше надо аккуратно. Наше существующее и единственное решение разобъём на два слагаемых $u_h = y_h + v_h$. При этом $L_h y_h = 0$, $y_h|_{t=0} = \phi_h$, $y_h\big|_{x=0} = \mu_1$, $y_h|_{x=X} = \mu_2$ и $L_hv_h = f_h$, $y_h|_{t=0} = y_h|_{x=0,X}=0$.

Берём $y$. С ним будет посложнее
\[
  \|y^{n+1}\|\le \max\big(\|\mu_1\|,\|\mu_2\|,\|y^n\|\big) \le\max\big(\|\mu_1\|,\|\mu_2\|,\|y^{n-1}\|\big) \le \max\big(\|\mu_1\|,\|\mu_2\|,\|\phi_h\|\big).
\]
Теперь для $v$. Тут совсем оценка приятная
\[
  \|v^{n+1}\|\le \|v^n\| + \tau \|f_h\| \le \|v^{n-1}\| + 2\tau \|f_h\|\le \dots\le \|v^0\| +  \RY k0n\tau\|f_h\| \le T\|f_h\|.
\]
Значит, схема устойчива.
\end{Proof}

\subsection{Устойчивость неявной схемы}
Вот такая схема
\[
  \frac{u_m^{n+1} - u_m^n}{\tau} = \frac{ u_{m-1}^{n+1} - 2u_m^{n+1} + u_{m+1}^{n+1}}{h^2} + f_m^{n+1}.
\]

Хотим показать, что схема безусловно устойчива, то есть неважно, как именно мы выбираем шаги.

Запишем схему таким же образом. 
\[
  u_m^{n+1}  + \frac{\tau}{h^2} (-u_{m-1}^{n+1} + 2 u_m^{n+1} - u_{m+1}^{n+1}) = u_m^n + \tau f_m^{n+1}.
\]
Ясно, что $\|u^{n+1}\| = |u_{m_0}^{n+1}|$, $m_0 $ "--- первый встретившийся номер, на котором достиглась норма.

Если $m_0 = 0$ или $M$. Тогда выполнено
\[
  \|u^{n+1}\| \le \max\big(\|\mu_1\|,\|\mu_2\|,\|u^n\| + \tau\|f_h\|\big)
\] 
И в этом случае мы всё доказали.

Если норма достигается внутри. Тогда внутри скобки по модулю $u_{m_0}^{n+1}$ самая большая. И знак скобки определяетс знаком этого эдемента. Пусть у нас $m=m_0$.
\[
  \|u^{n+1}\| = |u_{m}^{n+1}| \le \bigg| u^{n+1}_m+\frac{\tau}{h^2} (-u-^{n+1}_{m-1} + 2u_m^{n+1} - u_{m+1}^{n+1})\bigg|\le \|u^n\| + \tau\|f_h\|.
\]
Дальше повторяем тот кусок доказательства. И устойчивость доказана.

\subsection{Скорость сходимости}
Скорость сходимости будет такая же, как порядок аппроксимации. В явной схеме мы будем медленно шагать, вынуждены. Но в неявной схеме, чтобы была сходимость с тем же порядком, что и аппроксимация, нам необходимо и тут тоже $\tau\sim h^2$. Соответственно будем медленно считать.

Надо сооружать новую схему. Прежде чем это делать, соорудим оператор второго дифференцирования.
\subsection{Оператор второго дифференцирования}
Рассмотрим пространство сеточных (на отрезке $Mh = X$) функций $v = (0,v_1,v_2,\dots,v_{M_1},0)$ таки, что на границе нули. На этом пространстве определим оператор численного дифференцирования
\[
  \Lambda v_m := \Lambda v\big|_m = \frac{ v_{m-1} - 2v_m + 2_{m+1}}{h^2}, \quad m=1,\dots,M-1
\]
 Будем писать $\Lambda v = (0,\Lambda v_1,\dots, \Lambda v_{M-1},0)$.

Рассматриваем собственные значения оператора (будем рассматривать оператор с обратным знаком) и собственные функции $v^k\colon (-\Lambda) v^k = \lambda_k v^k$. Вообще говоря, индекс $k$ не может пробегать бесконечно много значений. Мы даже матрицу этого оператора выписывали. Мне удобно работать не на языке матриц, а на языке разностных уравнений.
\[
  \frac{-v_{m-1} + 2 v_m - v_{m+1}}{h^2} = \lambda v_m, v_0=v_M=0,\quad h = \frac XM.
\]
Надо найти $\lambda$ такие, что задача имеет ненулевые решения.
\[
  v_{m+1} - (2-\lambda h^2) v_m + v_{m-1} = 0,\quad v_0 = v_M = 0.
\]
Будем искать решение  виде $v_m = q^n$. Подставляем $q^2 - (2-\lambda h^2) q +1=0$.

Первый случай $D>0$. Тогда у нас два неравных вещественных корня $q_1\ne q_2\in \R$. Тогда $v_m = c_1 q_1^m + C_2 q_2^m$. Начинаем подставлять краевые условия $C_1+C_2 = 0$, $C_1 q_1^M + C_2 q_2^M = 0$. Отсюда получаем только нулевое решение.

Мы не оставляем надежду. Пусть $D=0$. Тогда $q_1=q_2 = q\in\R$. Решение должно выглядеть следующим образом $v_m = c_1 q^m + c_2 mq^m$. Опять подставляем всё, что можно: $C_1 = 0$. $C_2 Mq^M =0$. Но $q$ у нас точно не ноль, потому что по теореме Виета произведение корней единичка. И опять нулевое.

Придётся рассматривать $D<0$. Отсюда можно уже получить некую оценку на $\lambda$. Наше $q_{1,2} = \cos\phi\pm i\sin\phi$ (ведь произведение единичка). Должны были вы разбирать, что в этом случае решение $v_m = c_1 \cos m\phi + C_2\sin m\phi$. Подставляем $m=0$, тут же гибнет синус и $C_1=0$. Теперь подставляем дрожащей рукой $m=M$, получаем $C_2 \sin M\phi = 0$, причём $C_2$ равным нулю нас никак не устроит, значит, нулём должен быть синус $\sin M\phi = 0$. Отсюда $\phi = \frac{\pi k}M$.

Можно выписывать дискриминант, а можно попроще. А именно сумма корней в наших обозначениях $2\cos\phi$, а с другой по теореме Виета $(2-\lambda h^2)$. Значит
\[
  \frac{2 - \lambda h^2}{2} = \cos\frac{\pi k}M.
\]
Можно $\lambda$ заиндексировать
\[
  \lambda_k = \frac2{h^2} \left(1 - \cos\frac{\pi k}M\right) = \frac4{h^2} \sin^2\frac{\pi k}{2M}.
\]
На $k$ причём пока ограничений нет. Но задача была об операторе на конечномерном пространстве. Значит, либо какие-то $k$ не годятся, либо при некоторых $k$ значения повторяются.

Пусть $k=0$. Тогда дискриминант равен нулю, и мы этот случае забраковали. Значит, $k\ne 0$.

Дальше $k=1,\dots,k=M-1$ нам подходят. А уже $\lambda_M$ опять даёт нулевой дискриминант. Дальше идёт некое отражение $\lambda_{M+1} = \lambda_{M-1}$, $\lambda_{M+2} = \lambda_{M-2}$ и так далее.

Ладно, а что за собственные функции? Это решения $v^k\colon v^k_m = \sin\frac{\pi k m}{M}$. Здесь $m=1,\dots,M-1$.
Кстати автоматом получаем, что и при $m=0,M$ эти функции в наше пространство годятся.


\subsection{Возвращаемся к задаче теплопроводности}
Предлагается построить вот такую разностную схему
\[
  \frac{ u_m^{n+1} - u_m^n}{\tau} = \sigma \Lambda u_m^{n+1} + (1-\sigma) \Lambda u_m^n + f_m^n.
\]
Здесь $\sigma\in[0,1]$. При $\sigma=0$ получается явная схема, которую мы рассматривали, при $\sigma=1$ неявная.

Сейчас мы будем играться с аппроксимацией. Мы хотим получить аппроксимацию по $\tau$. Только аппроксимацию относительно точки $(m,n+0.5)$. Есть надежда, что при $\sigma=0.5$ как-то что-то сбалансируется.

Сейчас надо будет вспоминать, что мы выписывали в первом семестре про главные члены погрешности формул численного дифференцирования. У нас будет $(x,t) = (mh,n\tau)$. Будем выписывать аппрроксимацию на решении (на решении менее громоздко)
\[
  \frac{u(x,t+\tau) - u(x,t)}{\tau} = \CP ux\bigg|_{(x,t)} + \frac{\tau^2}{24} \CP{^3u}{t^3}\bigg|_{(x,t)} + \dots
\]
Теперь посмотрим на второе дифференцирование
\begin{multline*}
  \Lambda u \bigg|_{\left(x,t+\frac\tau2\pm\frac\tau2\right)} = \Lambda u\bigg|_{(x,t+\frac\tau2)} \pm\frac\tau2 \Lambda \CP ut\bigg|_{(x,t+\tau/2)} + O(\tau^2) = \\=
  \CP{^2u}{x^2}\bigg|_{(x,t+\tau/2)} + \frac{h^2}{12} \CP{^4u}{x^4}\bigg|_{(x,t+\tau/2)} \pm \frac\tau2 \Lambda \CP ut\bigg|_{(x,t+\tau/2)} + O(\tau^2,h^4).
\end{multline*}

Мы хотим оценить такую разность
\[
  \big(L_h[u] - f_h\big)\Big|_{(m,n+0.5)} = \bigg( f\big(mh,(n+0.5)\tau\big) - f_m^n\bigg) + \tau(\sigma-0.5)\Lambda\CP ut\bigg|_{(x,t+\tau/2)} + U(\tau^2,h^2).
\]

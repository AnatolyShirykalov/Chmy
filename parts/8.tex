\section{Полулекция перед Коробейниковым}

В прошлый раз вывели три простейших формулы численного интегрирования. Когда равномерная сетка по отрезку и берётся интеграл от интерполяционного многочлена. 
\[
  \int\limits_a^b f(x)\,dx\approx \RY j1n c_j f(x_i).
\]

Если $n=1$, то $x_1 = \frac{a+b}2 = \ol x$.

Если $n\ge2$, $x_1=a$, $x_n=b$, $x_j- x_{j-1} = \const$.

Обозначали $b-a=h$.
\[
  R(f) = I(f) - S(f) = \equiv \int\limits_a^b f(x)\,dx - \RY j1n c_j f(x_j).
\]
Для $n=1$ имеем $R(f) = \frac{h^3}{24} f''(\ol x) + O(h^5)$.

Для $n=2$ формулу назвали формулой трапеций и
\[
  R(f) = -\frac{h^3}{12} f''(\ol x) + O(h^5).
\]
Для $h=3$  получили формулу Симпсона и
\[
  R(f) = \frac{-h^5}{2\,880} f^{(4)}(\ol x) + O(h^7).
\]

\subsection{Простейшие составные квадратурные формулы}
Давайте поступим следующим образом. Наш отрезок $[a,b]$ разобьём на некоторое количество отрезков уже малой длины. Обозначим $x_1=a$, $x_n=b$. Разбиваем так, чтобы $x_{i+1}-x_i = h$. Теперь весь интеграл разбиваем
\[
  I_(f) = \RY j1{n-1}I_j(f) = \RY j1{n-1}\int\limits_{x_j}^{x_{j+1}} f(x)\,dx.
\]
Обозначим $\ol x_i = \frac{x_i + x_{i+1}}2$. И перейдём к составлению квадратурных формул.

Для формулы средних прямоугольников
\[
  I(f) = \RY j1{n-1} hf(\ol x_j) + R^\Pi_j(f) = \RY j1{n-1} hf(\ol x_j) + \frac{h^3}{24}\RY j1{n-1}\big(f''(\ol x_j) + O(h^2)\big).
\]
Пусть $\big|f''(x)\big|\le M_2$. Можем просуммировать
\[
  \big|R^\Pi(f)\big| \approx \frac{h^2}{24} M.
\]
Можно по-другому оценить. Сказать, что
\[
  R^\Pi(f)\approx \frac{h^2}{24} \RY j1{n-1} hf''(\ol x_j)\approx
\]
Это похоже на квадратурную формулу средних прямоугольников
\[
  \approx \frac{h^2}{24}\int\limits_a^b f''(x)\,dx.
\]

Дальше можем посмотреть на формулу трапеции.
\[
  I(f) = \RY j1{n-1}\frac h2\big(f(x_i) + f(x_{i+1})\big) - \frac{h^3}{12} \RY j1{n-1} f''(\ol x) + O(h^5) =
  h\left( \frac12 f(x_1) + f(x_2) + \dots +f(x_{n-1}) + \frac12f(x_n) \right) - \frac{h^2}{12} \int\limits_a^b f''(x)\,dx + O(h^4).
\]
В качестве упражнения остаётся написать составную квадратурную формулу для формулы Симпсона.

Для вычисления погрешности можно придумать аналог правила Рунге. Можно сделать шаг вдвое меньше, формулы вычесть и так далее.

Надо мысль как-то развивать. Если функция не дифференцируемая, такая техника у нас не пройдёт. А если больше узлов, как всё устраивать? Надо как-то обобщать. Пока приёмы индивидуальны. Попробуем получить более грубую оценку погрешности, но чтобы оценка работала для любого типа формул. Ну и давайте избавимся от необходимости ограниченной второй производной. Что делать, если этого нет. Как быть, если функция не достаточно гладкая. Если проблема носит искусственный характер (конечное число плохих точек), ну разбили на кусочки. Если $\sqrt{x}$ уже хуже.

Положим, что подынтегральную функцию можно разбить на два множителя 
\[
  \int\limits_a^b f(x)p(x)\,dx
\]
Здесь $f(x)$ гладкая, а все особенности лежат в $p(x)$.  Квадратурную формулу подбираем следующим образом
\[
  \int\limits_a^b f(x)p(x)\,dx \approx \RY j1n c_j f(x_j).
\]

Можно снова сделать то же самое, что мы только что делали. Во всех наших формулах добавятся значения $p(x)$ в узлах. Коэффициенты при симметричных узлах снова будут равны, если весовая функция симметрична. Иначе не будет. Как здесь оуеним погрешность. Наши способы не пройдут. Получим более грубую оценку. Хотим чтобы погрешность выражалась в терминах $b-a$, то есть длины отрезка. Идея будет следующая. Попытаемся получить единый результат на все формулы. Такая нужна какая-то характеристика формулы. Максимальная степень многочлена, для которого формула точна. Будем считать, что формула $S(f)$ верна для многочленов степени до $m$ включительно.

\section{Лекция какая-то}
В прошлый раз мы закончили правилом Рунге. Давайте теперь попытаемся изыскать внутренние резервы наших методов численного интегрирования.

Представили функцию $\int\limits_a^b f(x)p (x)\,dx \approx \RY j1n c_j f(x_j)$.

Неизвестных у нас здесь $2n$ штук. Подставлять мы будем многочлены $1,x,x^2,\dots,x^m$, их $m+1$ штука. Неизвестные $c_j$ и $x_j$, их $2n$ штук. Количество решений бывает всякое. Мы выдвигаем гипотезу, что у нас получится $m=2n-1$, то есть решение будет единственно. Ну только вот кто сказал, что когда мы решим, получим, что $x_j$ попадут в отрезок $[a,b]$.

Наложим ограничения на весовую функцию $p(x)$. Пусть она больше нуля почти всюду на нашем отрезке.
Ну и желательно, чтобы сама она была интегрируема.

Давайтпе обсудим, как квадратурная формула будет строиться. Я буду кое-что давать без доказательства. Будете шуметь, с доказательством дам.

Предположим мы формулу многочлена степени $2n+1$ построили. Если бы она существовала, каким свойством должны обладать её узлы?
\[
  \forall\ Q_{2n-1}(x)\pau \INTAB Q_{2n-1}(x) p(x)\,dx = \RY j1n c_jQ_{2n-1}(x_j).
\]
Пусть эта формула верна для многочленов до степени $2n-1$ включительно и мы знаем узлы этой формулы $x_j$. Тогда
\[
  \INTAB \omega_n(x) P_{n-1}(x) p(x)\,dx = 0,
\]
где $\omega_n(x) = \prod\limits_{j=1}^n(x-x_j)$.

Давайте смотреть $\omega_n(x) = P_{n-1}(x) = Q_{2n-1}(x)$, $Q_{2n-1}(x_j) = 0$.

Что это означает. Что $\omega_n$ ортогональна всем многочленам степени $n-1$.

Если есть вес $p(x)>0$, то существует (по Грамму"--~Шмидту) ортогональная система многочленов, перпендикулярную $p(x)$. Главное, что любой из этих многочленов имеет ровно $n$ корней и все они пребывают на $[a,b]$. Поэтому с точностью до множителя, это $\omega_n(x)$.

Поэтому в качестве узлов, мы обязаны брать нули соответствующего ортогонального многочлена.

\subsection{Пример}
Пусть $p=1$ "--- вес единичка. Тогда соответствующий ортогональный многочлен на $[-1,1]$, многочлен Лежандра
\[
  \psi_n(x) = \frac{1}{2^n n!} \DP{^n}{x^n} \big((x^2-1)^n\big).
\]

Пусть теперь
\[
  p = \frac{2}{\pi\sqrt{1-x^2}}.
\]
Тогда берём $\psi_n(x) = T_n(x)$ "--- многочлен Чебышёва.
\subsection{Формулы для коэффициентов}
Ну хорошо, а как быть с коэффициентами.

Итак сначала мы находим $x_j$ из уравнения
\[
  \int\limits_a^b \omega_n(x) P_{n-1}(x) p(x)\,dx.
\]

Далее подбираем $c_j$ таким образом, чтобы формулы была точной для многочленов степени до $n-1$ включительно. То есть
\[
  \int\limits_a^b f(x) p(x)\,dx\approx \int\limits_a^b L_n(x) p(x)\,dx = \sum c_j f(x_j)
\]
должно быть нашей итоговой формулой. Сейчас мы попробуем это доказать. Здесь $L_n$ "--- интерполяционный многочлен Лагранжа.

Возьмём произвольный $Q_{2n-1}(x)$. Поделим его с остатком на $\omega_n(x)$.
\[
  Q_{2n-1}(x) = \omega_n(x) g_{n-1}(x) + r_{n-1}(x).
\]
Что есть погрешность. На остатке она ноль, а не первом слагаемом уже обсуждали
\[
  R\big(Q_{2n-1}(x)\big) = \underbrace{R\big(\omega_n(x) g_{n-1}(x)\big)}_{=0}  + \underbrace{R\big(r_{n-1}(x)\big)}_{=0}.
\]
Мы постоили квадратурные формулы наибольшей алгебраической точности. Они называются квадратурными формулами Гаусса.


Какие свойства.
\begin{Ut}
  Если $p>0$ и $p$ "--- симметрична относительно $\frac{a+b}{2}$, то
\begin{enumerate}
\item $x_j$ симметричны;
\item $c_{n+1-j} = c_j$.
\end{enumerate}
\end{Ut}
Эти два свойства доказываются противно. Мы их доказывать не будем. Докажем третье.
\begin{Ut}
  Пусть $p>0$. Тогда $c_j>0$.
\end{Ut}
\begin{Proof}
Это свойство доказывается просто примером. Зафиксируем какое-то $k\in\{1..n\}$. Построим 
\[
f_k(x) = \left( \frac{\omega_n(x)}{x-x_k} \right)^2
\]
Это многочлен степени $2n-1$, значит, наша формулы для него точна
\[
  0< \INTAB f_k(x)p(x)\,dx = \RY j1n c_j f_k(x_j) = c_k \underbrace{f_k(x_k)}_{>0}.
\] 
\end{Proof}

В качестве упражнения можете выписать оценки погрешности. Грубо говоря, в том, что мы делали, заменить $n$ на $2n-1$.
Есть такая тактика провоцировать экзаменатора на сложные вопросы. Писать в билете всё, ктоме убойного вопроса, который вы знаете. Слишком скрытничать нельзя. Но какой-то изюм лючше выкладывать не сразу. Вас спрашивают этот сложный вопрос, думая, что вы не знаете, а вы сходу пишете. Обычно на этом все довольны, всё заканчивается. К сожалению, понимание такой тактики приходит не сразу.

\subsection{Двойные интегралы}
Мы ограничимся случаем прямоугольника за нехваткой времени. Давайте-ка возьмём плоскость $(x,y)$. На нём у на будет прямоугольничек $[a,b]\times [\alpha,\beta]$. Будем считать такой интегральчик
\[
  I(f) = \INT_\alpha^\beta\INTAB f(x,y)\,dx\,dy.
\]
Будем разбивать прямоугольник на много маленьких. Будем считать, что у нас уже стороны прямоугольника есть малые. Давайте через $S$ я обозначу $S = (b-a)(\beta-\alpha)$, $\ol x = \frac{a+b}{2}$, $\ol y = \frac{\alpha+\beta}{2}$. Простейший способ считать интеграл, первая простейшая формулу.
\[
  I(f)\approx S f(\ol x,\ol y).
\]

Понятно, что если мы будем иметь какую-то большую область-прямоугольник, мы можем разбить на много маленьких равных прямоугольников. По вертикали $M$ штук, по горизонтали $N$ штук. Тогда
\[
  I = \sum\limits_{i,j} S f(\ol x_i,\ol y_j).
\]

Локальная формула должны быть точной для линейных функций.
\[
  f(x,y) = f(\ol x,\ol y) + (x-\ol x) f'_x + (y-\ol y)f'_y + \frac12 (x-\ol x)^2 f''_{xx} + \frac12 (y-y\ol y)^2f''_{yy} + (x-\ol x)(y-\ol y) f''_{xy} + \dots
\]
Ну вот давайте подставлять
\[
  I(f) = S f(\ol x,\ol y) + \frac1{24}S \left( (b-a)^2f''_{xx} + (\beta-\alpha)^2f''_{yy} \right) + \dots.
\]

Теперь берём общий случай, когда $b-a$ и $\beta-\alpha$ не являются малыми параметрами. Будем делить на маленькие прямоугольники. $N$ по горизонтали, $M$ по вертикали. Для каждого элементарного прямоугольника погрешность с точностью до малых более высокого порядка будет записываться следущим образом
\[
  R_i(f)\approx \frac1{24} S_i\left( \left( \frac{b-a}{N} \right)^2f''_{xx} + \left( \frac{\beta-\alpha}{M} \right)^2f''_{yy} \right).
\]

Отсюда $R(f) = \sum\limits_i R_i = O\left( N^{-2} + M^{-2} \right)$. Хорошо это или плохо? Мы можем работать как и в одномерном случае. Можем придумывать правило Рунге, дробить больше сетки. 
Также можем уточнять локальные формулы.

Давайте сделаем себе жизнь потяжелее. Займёмся последовательным интегрированием.
\[
I(f) = \INT_\alpha^\beta \INTAB f(x,y)\,dx\,dy = \INT_\alpha^\beta F(y)\,dy,\quad
 F(y) = \INTAB f(x,y)\,dx.
\]

Пусть $I\approx \sum\limits_j \ol c_j F(y_j)$. Теперь для вычисления каждого и $F(y_j)$ строим формулы
\[
  F(y_j)\approx \sum\limits_i c_i f(x_i,y_j).
\]
Теперь собираем это вместе
\[
  I\approx \sum\limits_{i,j} c_i\ol c_j f(x_i,y_j).
\]
Ну допустим мы с вами взяли большой прямоугольник. Разбили его с шагами $h_x$ и $h_y$. И по каждой стороне взяли составную формулу трапеций. Что в результате получится. Тогда
\[
  \frac{c_i\ol c_j}{h_x h_y} = \begin{cases}
  1 & \text{внутренняя точка}\\
  \frac12 & \text{внутренная только по одной переменной}\\
 \frac1n & \text{угловая}
\end{cases}.
\]

Отсюда получим, что $R(f) = O \left( h_x^2 + h_y^2 \right)$.
Ну сюда можно любые наши формулы одномерные перетащить.

Поверьте мне, что вычисление интегралов в десятимерном пространстве "--- это актуально. Когда стали их считать, столкнулись с совершенно неожиданными проблемами. Предположим мы считаем по составной формуле трапеций или даже по составной формуле средних прямоугольников.
\[
  I(f) = \underbrace{\INT_0^1\dots\INT_0^1}_{10} f dx_1\,\dots\,dx_{10} \approx \RY j1n c_j f(x_j).
\]
Пусть у нас есть $\e>0$. Тогда $R(f) = O(N^{-2})$. Осюда $N^{-2}\approx \e$ и $N\sim \frac1{\sqrt{\e}}$.

Пусть $\e = 10^{-4}$. Тогда $N=10^2$. Но это количество делений по одной грани. Ну тогда $n=N^{10}$. Это мы подставляем дрожащей рукой.

Вот для двумерного всё было хорошо, для десятимерного всё плохо. А где-то посередие есть граница.

Решение этой проблемы называется методом Монте-Карло. На экзамене нужно будет описать проблему и выписать те соотношения, которые я вам сейчас дам. Я дам вам в общих чертах.

Будем считать, что живём в нашем кубе или области, не важно. Пусть $\Omega = [0,1]^k$. Сгенерируем $N$ штук попарно независимых точек $P_j\in \Omega$. (То есть попарно независимыми генераторами.) В качестве квадратурной формулы возьмём вот такую штуку
\[
  S_N(f) = \frac1N \RY j1N f(P_j).
\]
А дальше имеет место вот такое соотношение
\begin{The}
 С вероятностью $(1-\theta)$ выполнено такое неравенство
\[
 \big| I(f) - S_n(f)\big| \le \sqrt{\frac{D(f)}{\theta N}},\quad
 D(f) = I(f^2) - I^2(f),\quad \theta\in(0,1).
\]
\end{The}

На следующей лекции займёмся численными методами алгебры.

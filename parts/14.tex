\section{3 декабря}
Никак вас не отучу вставать. У нас итерационные методы решений линейных уравнений. Наш метод приспособлен для системы $Ax = b$, где $A = A^T>0$. Было известно, что собственные значения жили на отрезке
\[
  0< m \le \lambda(A)\le M<\infty.
\]
Метод простой итерации
\[
  x^{n+1} = x^n - \alpha (A x^n -b).
\]
Было доказано, что оптимальное значение вот такое
\[
  \alpha = \frac2{M+m}.
\]
Если обозначить $z^n = x^n -\ol x$, где $\ol x $ "--- решение, то $\|z^n\|_2\le q^n_0\|z^0\|_2$, где $q_0 = \frac{M-m}{M+m}$.

Почему это надо улучшать. Наши предположения о точных оценках $m$ и $M$ несут проблемы. Что такое $q_0$
\[
  q_0 = \frac{\frac Mm - 1}{\frac Mm+1}.
\]
А дробь $\frac{M}m$ есть оценка обусловленности матрицы. Если матрица плохо обусловлена, то $q_0$ близка к единицы.

Если 
\[
  A = \begin{pmatrix}
 2 & -1 & 0 & \dots\\
 -1 & 2 & -1 & \\
 0 & \ddots & \ddots & \ddots
\end{pmatrix}
\]
Здесь $\cond_2(A) = O(N^2)$. Скорость сходимости будет такая, что ошибки вычислений на шаге не будут давать вообще методу сходиться.

Итак нам надо получить метод с более высокой скоростью сходимости. А также нужно смягчить входные данные. Ослабить условия на $m$ и $M$.

Итак, что можно улучшить. Что здесь хромает. Что везде $\alpha$ одно и то же. За $n$ шагов у нас будет $n$ параметров теперь.
\[
  x^{n+1} = x^n - \alpha_{n+1}(A x^n - b);\qquad
  \ol x = \ol x - \alpha_{n+1}(A\ol x - b).
\]
Сразу ситуация ухудшится. Мы теперь не можем вычесть из одного уравнения другое. Раньше при вычитании вылезало $z^n$, а теперь
\[
  z^{n+1} = (E - \alpha_{n+1} A)z^n.
\]
Отсюда
\[
  z^n = \prod\limits_{j=1}^n(E - \alpha_j A)z^0 = P_n(A)z^0.
\]
Получили последовательность матричных многочленов $P_n(A)$. От которых нам нужно лишь то, чтобы их норма стремилась к нулю. Какую норму брать? Опять удобнее будет взять вторую норму.

Как сформулировать задачу. По-прежнему минимаксная.
\[
 \|P_n(A)\|_2 = \max\limits_{\lambda(A)} \big|P_n\big(\lambda(A)\big)\big|\le \max\limits_{m\le \lambda\le M} \big|P_n(\lambda)\big|.
\]

Имеем
\[
  P_n(\lambda) = \prod\limits_{j=1}^n(1-\alpha_j\lambda)
\]

Вспоминаем, что такое многочлены Чебышёва
\[
  T_n(x) = \begin{cases}
  \cos(n\arccos x),& |x|\le 1;\\
  \frac{(x+\sqrt{x^2-1})^n + (x - \sqrt{x^2-1})^n}{2},& |x|>1.
\end{cases}
\]

При этом отрезок $[m,M]$ переведём в $[-1,1]$. Тогда $\lambda = \frac{M+m}{2} + \frac{M-m}{2}$. Отсюда выражаем $x$. Нужно только правильную нормировку сделать
\[
  P_n(\lambda) = \frac{T_n\left( \frac{2\lambda - (M+m)}{M-m} \right)}{T_n\left( -\frac{M+m}{M-m} \right)}
\]
Можно вернутся к нашему доказательсву того, что приведённые многочлены Чебышёва наиболее близкие к нулю в соответствующем классе. Оно сюда перетаскивается.

У нас задача ставится так. Надо подобрать набор $\{\alpha_j\}_{j=1}^n$. Совпадение корней не даёт совпадение многочленов. Но мы уже отнормировали как надо. Итак, какие корни
\[
  \cos(n\arccos x) = 0\imp n\arccos x = -\frac\pi2 + \pi m\imp x_m = \cos\frac{\pi(2m-1)}{2n},\quad m=1,\dots,n.
\]
Надо в терминах $\lambda$ написать. Нам надо, чтобы $P_n(\lambda_j) = 0$. То есть $\lambda_j = \frac{M+m}2 + \left( \frac{M-m}2 \right)\cos\frac{\pi(2j-1)}{2n}$. Мы хотим, чтобы это были корни нужного нам многочлена. Таким образом
\[
  \alpha_j = \frac1{\lambda_j}.
\]

Итого, если воспользоваться положительной определённостью матрица $A$ и границами спектра, то предлагаемый метод является лучшим. Эту задачу нельзя решить лучше.

Итак, $\|P_n(A)\|_2\le \left|\frac{T_n\big(x(\lambda)\big)}{T_n\big(x(0)\big)}\right|\le \frac1{\Big|T_n\big(x(0)\big)\Big|}$.

Я напомнб, что $x(\lambda) = \frac{2\lambda - (M+m)}{M-m}$.

Имеем $\big|x(0)\big|>1$. Осюда $T_n\big(x(0)\big) = \frac{\left( x(0) + \sqrt{ x(0)^2 - 1} \right)^n + \left( x(0) - \sqrt{x(0)^2 -1} \right)^n}2$. При этом $x(0) = -\frac{M+m}{M-m}$.

Мы должна написать такую вещь
\[
  -\left( \frac{M+m}{M-m} \right)\pm \sqrt{ \left(\frac{M+m}{M-m}\right)^2-1} = \frac{-(M+m)\pm 2\sqrt{Mm}}{M-m} = 
 - \frac{\left( \sqrt{M}\mp\sqrt{m} \right)^2}{M-m} =
 \left[
  \begin{aligned}
    -\left( \frac{\sqrt{M} - \sqrt{m}}{\sqrt{M}+\sqrt{m}} \right) = -q_1;\\
    -\left( \frac{\sqrt{M} + \sqrt{m}}{\sqrt{M}-\sqrt{m}} \right) = -\frac1{q_1}.\\
  \end{aligned}
\right. 
\]

Таким образом есть два случая, но в любом выходит либо $-q_1$, либо $-\frac1{q_1}$.
\[
  q_1 = \frac{\sqrt M - \sqrt m}{\sqrt M - \sqrt m}.
\]

Отсюда 
\[
  \big\|P_n(A)\big\|_2\le \frac{1}{\left|\frac{(-1)^n}2(q_1^n + q_1^{-n})\right|} = \frac{2q_1^n}{1 + q_1^{2n}}<2q_1^n.
\]

Получили такой метод. $\alpha_j = \frac1{\lambda_j}$, $j=1..n$. А погрешность
\[
  \|z^n\|_2 < 2 q_1^n \|z^0\|_2.
\]
Гораздо лучше. Но всё равно плохо для той трёхдиагональной матрицы.

Итерационный метод с Чебышевским набором параметров
Метод Ричардсона.

Получился такой итерационный метод
\[
  x^{n+1} x^n - \lambda_{n+1}(Ax^n - b).
\]
К сожалению он работает по совокупности. Выбираем $x^0$, выбираем $n$. Находим $x^1,x^2,\dots,x^n$. Тогда
\[
  z^n = \prod\limits_{j=1}^n (E - \alpha_j A) z^0.
\]
Именно множитель будет мал по норме.

Но если мы хотим сделать следующий шаг, то у нас $\alpha_{n+1}$ уже нет.

Вообще $z^n$ мы вичислить не можем. $z^n = x^n - \ol x$, а $\ol x$ мы не знаем.
\[
  \|z^n\|_2 < 2 q_1^n\|z^0\|_2.
\]

Что мы можем сделать. Найти номер $n$, для которого $2 q_1^n\le \e$. Но на практике может вылести всё что угодно.

Но часто смотрят в терминах невязки $\|r^n\| = \|Ax^n - b\|$.

Можно даже на каждом шаге смотреть на невязку. Если на $n$ шаге она нас не устраивает, то делаем ещё раз, но уже в качестве $x^0$ берём то, что получилось на $n$-й операции.


Ещё есть проблема вычисления слишком больших чисел, которые не влезают в машинную арифметику. Когда мы вычисляем всё подряд, то есть $\alpha_j$ подряд, по тем формулам, которые я выписывал $\alpha_j = \frac1{\lambda_j}$, $j=1..n$. Мы знаем, что
\[
  \prod\limits_{j=1}^n(E-\alpha_jA)
\]
по норме меньше единицы. Но это не значит, что всякая меньше единицы. Там могут быть несколько множителей, которые больше единицы по норме. И асимптотически всё хорошо. Но может быть шаг, где накопленная погрешность перевалит за максимально допустимую границу.

Будем перемешивать. Приведу алгоритм без доказательства.
\subsection{Алгоритм перемешивания}
Нам надо перемешать числа $1,2,\dots, n$, для простоты пусть $n = 2^p$. Всё делаем по шагам. Берём последовательно $k=1,\dots, p$. И каждый раз будем перемешивать числа $1,2,\dots,2^k$.

Пусть $k=1$. Тогда $\{1,2\}$ будет результатом перемешивания (тут не важно, но что-то надо выбрать).

Пусть для $(k-1)$ мы уже все $1,\dots,2^{k-1}$ перемешали в $\{b_1^{(k-1)},\dots, b_{2^{k-1}}^{(k-1)}\}$. Как тогда перемешать для $k$. Число элементов увеличивается в два раза. Делаем следующее
\[
  \{b_1^{(k-1)},2^k+1-b_1^{(k-1)}, b_2^{(k-1)}, 2^k+1 - b_2^{(k-1)},\dots\}
\]

Для $16$ имеем $\{1,16,8,9,4,13,5,12,2,15,7,10,3,14,7,11\}$.

Итак у нас есть два алгоритма для решения системы. Они сходились как геометрическая погрессия с показателями
\[
  q_0 = \frac{M-m}{M+m} \sim \frac{\cond_2(A) - 1}{\cond_2(A) + 1};\qquad
  q = \frac{\sqrt{\cond_2(A)} - 1}{\sqrt{\cond_2(A)} + 1}.
\]

\subsection{Другая норма}
Пусть имеется матрица $A = A^T>0$. Введём матрицу $A^{\frac12}$. Попытаемся определить её таким образом $A^{\frac12}A^{\frac12} = A$. Существует ли такая?

Если матрица $A= A^T>0$, то в принципе существует ортогональная $Q$, для которой $A = Q \Lambda Q^T$, Причём $\Lambda = \diag(\lambda_1,\dots, \lambda_N)$. Другое дело, что в общем случае конструктивно построить матрицу $Q$ нельзя, но она существует. Определим
\[
  \Lambda^{\frac12} = \diag(\sqrt{\lambda_1},\dots,\sqrt{\lambda_N}).
\]
Далее определим $A^{\frac12} = Q\Lambda^{\frac12}Q^T$.
Нам такую матрицу строить не придётся, но мы будем постоянно пользоваться её наличием.
Такая конструкция позволяет ввести своеобразную норму, которая оказывается очень эффективной.

Рассмотрим векторную норму $\|x\|_A = \sqrt{ (Ax,x)}$.
\begin{Ut}
  $\|x\|_A$ "--- норма. 
\end{Ut}
\begin{Proof}
Можно выписывать аксиомы и проверять. Но можно и по-другому. Для всякого $x$
\[
  \|x\|_A = \sqrt{ (A^{\frac12} A^{\frac12} x,x)}
\]
Я утверждаю, что корень конструкция симметричная. $(A^{\frac12})^T = A^{\frac12}>0$. Почему, ну распишите, через $\Lambda$ и $Q$. Тогда
\[
  \|x\|_A = \sqrt{(A^{\frac12} x,A^{\frac12}x)} = \|A^{\frac12}x\|_2.
\]
\end{Proof}

Норму называют энергетической. Когда нет информации о границе спектра, её и используют.
\subsection{Без границ спектра}
Есть система $Ax = b$, $A= A^T>0$. Будем строить итерационный метод такого вида
\[
  x^{n+1} = x^n - \alpha_{n+1} (A x^n - b).
\]
Из каких соображений будем его строить? Пусть точное решение $\ol x = \ol x - \alpha_{n+1}(A\ol x- b)$. Вычтем, получим погрешность
\[
  z^{n+1} = (E - \alpha_{n+1} A)z^n.
\]

Предположим мы добрались $x^n$, как по нему построить $\alpha_{n+1}$, чтобы $\|z^{n+1}\|\to \min$, то есть следующий шаг мы хотим сделать наилучшим образом.

\subsection{Неудачная попытка}
Сейчас будет попытка, которая закончится неудачной. Попытаюсь без энергетической нормы всё сделать. Это объяснит, почему в итоге всё будет так замысловато.

У нас есть $z^{n+1} = (E - \alpha_{n+1} A)z^n$. Возведём это в скалярный квадрат.
\[
  (z^{n+1},z^{n+1}) = \big(E - \alpha_{n+1} A)z^n,E - \alpha_{n+1} A)z^n\big).
\]
И раскрываем
\[
  \|z^{n+1}\|_2^2 = \|z^n\|_2^2 - 2\alpha_{n+1} (Az^n,z^n) + \alpha^2_{n+1}\|Az^n\|_2^2.
\]
Отсносительно $\alpha_{n+1}$ это парабола с ветвями вверх. Минимум находится в вершине.
\[
  \alpha_{n+1} = \left( \frac{-b}{2a} \right) = \frac{(Az^n,z^n)}{(Az^n,Az^n)}.
\]
Вопрос, а можем ли мы эту конструкцию посчитать?
\[
  Az^n = A(x^n-\ol x) = A x^n - A \ol x = Ax^n - b = r^n.
\]
Ну ничего. Отсюда
\[
  \alpha_{n+1} = \frac{(r^n,z^n)}{(r^n,r^n)}.
\]
Остался $z^n$ и мы упёрлись в тупик. Подход не прошёл, хотя идея была красивая.
\subsection{Теперь как надо}
Домножим $z^{n+1} = E - \alpha_{n+1} A)z^n$ на $A^{\frac12}$. У нас если одна матрица в умножении единичная, то порядок умноженмия не важен. А также $A^{\frac12}A = A A^{\frac12}$. Получается
\[
  A^{\frac12} z^{n+1} = (E - \alpha_{n+1}A)A^{\frac12}z^n.
\]
Возводим в скалярный квадрат
\[
  \|z^{n+1}\|_A^2 = \|z^n\|^2_A - 2\alpha_{n+1} (A A^{\frac12}z^n,A^{\frac12 z^n}) + \alpha_{n+1}^2 \|A A^{\frac12} z^n\|^2_2.
\]
Вершина параболы теперь
\[
  \alpha_{n+1} = \frac{-b}a = \frac{ (A A^{\frac12} z^n,A^nz^n)}{(A A^{\frac12} z^n,A A^{\frac12} z^n)} = \frac{(Az^n,Az^n)}{(Az^n,A^2z^n)} = \frac{(r^n,r^n)}{(r^n,Ar^n)}.
\]
О сходимости пока не говорю.

\section{Лекция 12}
Сегодня мы будем рассматривать ситуацию, когда матрица системы не является квадратной. Уже обсудили, что когда число уравнений меньше числа неизвестных, нам неинтересно. Вас уже на всех предметах этому обучали.

Интереснее, когда число уравнений больше числа неизвестных. Такая система может быть и совместной, но вообще говоря нет. 

\begin{Def}
Матрицы $A$ и $AB$ ортогонально подобны, если есть ортогональная $Q\colon QQ^T=E$, для которой $A = Q^T B Q$.
\end{Def}

\begin{Ut}
  У ортогонально подобный собственные значения одинаковые
\end{Ut}
\begin{Proof}
Доказательство очень простое
\[
  |A-\lambda E| = |Q^T B Q - \lambda Q^T T| = |Q^T (B-\lambda E) Q| = |Q^T| |B-\lambda E| |Q| = |B-\lambda E|.
\]
\end{Proof}

Отсюда QR-метод нахождения собственных значений. Доказательство очень сложно, я его рассказывать не буду.

Пусть матрица у нас простой структуры, то есть все собственные значения различны, $|\lambda_1| >\dots >|\lambda_n|$.
Тогда есть процедура превращения её в верхне треугольную, причём на диагонали будут собственные числа.

Положим $A^{(0)}=A$, $A^{(0)} = Q_0R_0$. Обозначим $R_0Q_0 = A^{(1)}$. Они, $A^{(0)}$ и $A^{(1)}$,  действительно ортогонально подобны. Дальше процедура очень простая
\[
  A^{(1)} = R_1 Q_1; A^{(2)} = Q_1 R_1.
\]
В матрице $A^{(k)}$ при $i>j$, то $|a_{ij}|\le c\left|\frac{\lambda_i}{\lambda_j}\right|^k$, значит, на диагонали с ростом $k$ будет что-то похожее на собственные числа.

\subsection{Переопределённые системы}

Чтобы их изучать, нужно знать сингулярное разложение матрицы.
Пусть у нас имеется $A$ размера $m\times n$, причём $m\ge n$.
\begin{The}
 $\forall\ A(m\times n), m\ge n$, $\rang A = r\pau \exists\ U(m\times m), V(n\times n)$ ортогональные, для которых
\[
  U^T A V = \Sigma
\]
Где $\Sigma$ матрица $m\times n$. Где сверху слева блок $\diag (\sigma_1,\dots,\sigma_r)$, остальные нули, а $\sigma_1\ge \dots \sigma_r$.
\end{The}

Доказательства не будет. Рассмотрим
\[
  \Sigma \Sigma^T = U^T A V V^T A^T U = U^T A A^T U.
\]
Это матрица $m\times m$. На диагонали квадраты сингулярных значений. Таким образом, сингулярные значния в квадрате, это собственные значения $A A^T$. С другой стороны
\[
  \Sigma^T \Sigma = V^T A^T U U^T A V = V^T A^T A V.
\]
Это уже матрица $n\times n$. Ненулевые собственные числа образуют одинаковый набор и в $AA^T$ и в $A^TA$. Вместо с нулевыми из разное количество.


Почему нельзя запускать $QR$ алгоритм для $AA^T$, чтоы найти сингулярные значения. Пример
\[
  A = \begin{pmatrix}
  1 & 1\\
  \beta & 0\\
  0 & \beta
\end{pmatrix}
\]
Пусть $\varepsilon>0$ и $\beta^2<\e<\beta$. То есть $\beta^2$ мы в наших вычислениях не видим, а $\beta$ вполне ощутима.
\[
  A^T A = \begin{pmatrix}
1+\beta^2 & 1\\
1 & 1+ \beta^2
\end{pmatrix}
\]
Если считать честно, то $\sigma_1(A)= \sqrt{2+\beta^2}$, $\sigma_2(A) = |\beta|$. Если теперь на этом этапе счесть $\beta^2$ малым, то $\sigma_1(A) = \sqrt{2}$, $\sigma_2(A) = |\beta|$.
Если бы мы сразу не видеть $\beta^2$, то матрица имеет вил
\[
  \begin{pmatrix}
1 & 1\\
1 & 1
\end{pmatrix}
\]
Тут одно из собственных значений ноль. А это уже нехорошо.


Пусть у нас есть некоторая матрица $A(m\times n)$, $m>n$, $\rang(A)=n$. Формально напишем задачу
\[
  Ax = b.
\]
В общем случае решать систему нельзя. Она имеет решение лишь в случае хорошего $b$, если $b$ лежит в пространстве, натянутом на столбцы матрицы $A$. Записывать эту систему мы не можем.

Введём вектор невязки $\ol r = b - Ax$. И подбором $x$ будем минимизировать его длину.
\[
 \|\ol r\| = \sqrt{\Sigma r_i^2}.
\]
Чтобы не мучиться, минимизируют не корень, а сумму квадратов.

У нас $\ol b = \ol b_1 + \ol b_2$, $b_1\in \Ln(A)$, $a_2\perp \ol a_i$.
Подберём $|hat x\colon (b-A\hat x)\perp \ol a_i$. Отсюда $A^T(\ol b - A\hat x) = 0$.
\[
  A^T A \hat x = A^T b.
\]
Это называется нормальной системой.

Теперь с обратной стороны к этому подойдём. Тут интересна сама техника, а не результат. Те же условия на матрицу $A$. Рассмотрим просто так нормальную систему 
\[
  A^T A \hat x = A^T b.
\]
Обозначим $\hat r = b - A \hat x$, и без крышки $r = b - Ax$. Хотим доказать, что
\[
  \|\hat r\|^2_2\le \|r\|^2_2\pau\forall\ x.
\]
Берём произвольный вектор $x$. По нему строим $r = b- Ax = b-A\hat x + A \hat x - A x$. Могу это немножечко причесать
\[
  r = \hat r + A(\hat x - x).
\]
Берём вторую норму в квадрате
\[
  \|r\|^2_2 = (r,r) = r^T r = \big(\hat r + A(\hat x- x)\big)^T\big(\hat r + A(\hat x- x)\big) = 
 \big(\hat r^T + (\hat x - x)^T A^T\big)(\big(\hat r + A(\hat x - x)\big).
\]
Но так как $\hat x$ есть решение нормальной системы, то $A^T\hat r = 0$. Раскрываем и учитываем
\[
  \|r\|^2_2 = \hat r^T\hat r + (\hat x - x)^T A^T A(\hat x - x) = 
  \|\hat r\|^2_2 + \big\|A(\hat x-x)\big\|^2_2\ge \|\hat r\|^2_2.
\]

Получили то же самое без геометрической интерпретации. Ещё тут единственность вылезла.

\subsection{Погрешность}
А как погрешность вычислений считать? Для квадратного случая у нас было понятие обусловленности матрицы. Тут попытаемся ввести что-то подобное.

Нам нужно какое-то подобие обратной матрицы.
\begin{Def}
Пусть $A(m\times n)$, $m\ge n$, $\rang(A)=n$. Тогда $A^+(n\times m)$ назовём псевдообратной к матрице $A$, если $A A^+ A = A$.
\end{Def}

Пусть $m=n$. Как можно определять обратную матрицу? Например через систему
\[
  Ax = b\imp x = A^{-1}b.
\]
Если $m>n$, мы можем писать нормальную систему $A^T A\hat x = A^T b$. Её решение
\[
  \hat x = (A^T A)^{-1} A^T b.
\]

Гипотеза такая: $A^+ = (A^T A)^{-1}A^T$.

Ну действительно $A(A^T A)^{-1} A^T A = A$.
В случае матрицы полного ранга есть единственность.

Будем рассматривать две задачи наименьших квадратов. Исходную и невозмущённую.
\[
  \|b - Ax\|_2\to\min,\quad \|\Til b - Ax\|_2\to\min.
\]
Матрица одна и та же, подпортили только правую часть. К ним нормальные системы
\[
  A^T A x = A^T b;\quad A^T A\Til x = A^T \Til b.
\]

Оказывается, всё не так просто. Разложим $b = b_1+b_2$, причём $x_2\perp \ol a_i$. То же самое с волнами $\Til b = \Til b_1 + \Til b_2$, $\Til b_2\perp \ol a_i$.

Имеет место следующая теорема
\begin{The}
$\frac{\|x-\Til x\|_2}{\|x\|_2}\le \cond_2(A) \frac{\|b_1-\Til b_1\|_2}{\|b_1\|_2}$, где
\[
  \cond_2(A) = \|A\|_2\|A^+\|_2.
\]
\end{The}
\begin{Proof}
  Икс есть решение нормальной системы
\[
  x = (A^TA)^{-1} A^T(b_1+b_2),
\]
причём $A^Tb_2 = 0$. Значит, $x = A^+ b_1$. Аналогично $\Til x = A^+ \Til b_1$. Мы получили, что
\[
  x- \Til x =A^+(b_1-\Til b_1).
\]
Отсюда
\[
  \|x-\Til x\|_2\le \|A^+\|\cdot \|b_1-\Til b_1\|_2.
\]

А как связаны $x$ и $b_1$. Это просто $Ax = b_1$. Это именно правильная решаемая система и решение наши иксы. Таким образом, $\|b_1\|\le \|A_2\|\|x\|$. Или, 
\[
\frac1{\|x\|}\le \frac{\|A\|_2}{\|b_1\|}.
\]
Вот мы всё и получили.
\end{Proof}

Забываем про псевдообратные конструкции. Решаем задачу
\[
  A^T A x = A^T b.
\]
Рассматриваем правую часть как цельную. Левая часть имеет матрицу $A^TA$. У неё есть обусловленность.
\begin{Ut}
  $\cond_2(A^TA) = \cond_2^2(A)$.
\end{Ut}
\begin{Proof}
Нам надо доказать, что $\|A^T A\| \|(A^TA)^{-1}\| = \|A\|\|A^+\|$.

 Что такое квадрат нормы матрицы $A$
\[
  \|A\|^2_2 = \max\limits_i \lambda_i(A^TA);\quad
  \|A^TA\|^2 = \max\limits_i \lambda_i(A^TA).
\]
Таким образом, $\|A\|^2_2 = \|A^T A\|_2$. Первую половину равенства мы доказали.

Попробуйте доказать, что $\|A\|_2 = \|A^T\|_2$.

И ещё в качестве задачи $A^+(A^+)^T = (A^TA)^{-1}$.

А отсюда следует, что $\|A^+\|^2_2 = \|(A^TA)^{-1}\|_2$.
\end{Proof}

\subsection{Как же её решать эту задачу МНК}
Единственный разумный алгоритм для больших размерностей и плохих обусловленностей, это построение сингулярного разложение.

\section{Новый сем}
Спасибо, что дождались. Так ну что, поздравляю, что преодолели сессию. Надеюсь, что все преодолели. У нас в конце семестра экзамен. У нас четыре лекции пропадают в праздничные дни. Буду ужимать материал. Обычно я в начале выдаю программу курса, теперь повременю. Сам не знаю, что ещё успею прочитать. Есть вещи, которые пропускать нельзя.

Сначала нелинейные системы, потом дифференциальные, потом некоторые виды уравнений в частных производных.
\section{Решение нелинейных уравнений и систем}
Бывают линейные, бывают квадратные. Бывают кубические, которые так-то тоже решаются. Остальные в общем виде не решаюся. Задачники по этому поводу: сборники исключений. А в общем виде, если у нас есть
\[
  f(x) = 0,
\]
явной формулы у нас с вами не будет.

Давайте договоримся, что мы должны сделать. Найти $x$, для которого $f(x) = 0$, этого недостаточно.

Раскусим случай, когда решений нет и когда решений бесконечно много.

Сузим задачу. Пусть имеется интервал $(a,b)$, есть $f(x)$. Будем искать алгоритм, который с заранее заданной точностью находит решение. А кратность корня? Многие алгоритмы отрабатывают хуже на кратных корнях, чем на простых.

Мы будем получать естественно некоторую последовательность $\{x_n\}$. Можно требовать сильную сходимость $x_n \to x$, а можно слабую $f(x_n) \to 0$. При этом необязательно стремление $x_n\to x$ при слабой сходимости.
% два рисунка

На интервале $(a,b)$ пусть имеется единственный корень уравнения $f(x)=0$. Будет рассуждать в терминах сильной сходимости. В идеале надо задать $\e>0$ и остановиться тогда, когда 
$|x_k-\ol x|<\e$.

Один из алгоритмов у вас сразу есть. Метод деления отрезка. Пусть $f(a)f(b)<0$, тогда есть хотя бы один корень, делим отрезок пополам. Выбираем половину, где меняется знак. В конце концов мы оказываемся на отрезке достаточно малой длины. Середину берём за ответ. Всё замечательно, кроме того, что алгоритм медленно сходится. Убывает как геометрическая прогрессия с множителем $\frac12$.
Основной плюс простота.
Если каждое вычисление функции дорогостоящая процедура, такое количество шагов может оказаться недопустимым.

Введём некий формализм. Чтобы получить приближение $x_{n+1}$, что надо сделать. От чего оно зависит? Самый простой вариант $x_{n+1} = \phi(x_n)$ "--- одношаговый метод или метод простой итерайии. Или же $x_{n+1} = \phi(x_n,x_{n-1},\dots,x_{n-k})$.

Начнём с одношагового. Если перейдём к пределу, вроде должны получить $\ol x = \phi(\ol x)$. То есть функцию $\phi$ нужно подобрать, а потобрать можно кучей способов. Как её строить? Само простое $x = x - g(x)f(x)$, где $f(x)\ne 0$ на $(a,b)$. Это не единственный способ, но самый простой. Уже здесь неимоверное число способов. Но достаточно ли мы потребовали? Надо, чтобы была сходимость.

Предположим, что $\phi$ достаточно гладкая. Мы хотим оценить $x_n-\ol x = \phi(x_{n-1}) - \phi(\ol x) = \phi(\xi_n) (x_{n-1} - \ol x) \hm= \phi'(\xi_n)\dots\phi'(\xi_1) (x_0-\ol x)$.
Если мы потребуем, чтобы $\big|\phi'(x)\big|\le q<1$, то $|x_n-\ol x| \le q^n |x_0-\ol x|$.

Можете меня укорить за то, что я поругал метод половинного деления. Но метод половинного деления вот так $x_{n+1} = \phi(x_n)$ не выписывается. Там две точки важны и вообще целая песня.

Как то, что я анонсировал, может выглядеть на практике. Пусть сначала производная $\phi$ бегает от нуля до единицы.
% рисунок
А как остановиться? $|x_{n+1}-x_{n}|<\e$ достаточно? Да кто его знает, может и нет. 

Пусть теперь производная $\phi$ от минус единицы до нуля. Тогда
% рисунок
Тут корень всегда между двумя соседними итерациями, то есть $\ol x \approx \frac{x_{k+1} + x_k}2$.

Как получить более высокую сходимость метода?

Под скоростью сходимости метода я буду понимать такую вещь
\[
  |x_{n+1} - \ol x| \le C |x_n - \ol x|^m.
\]
Если мы можем подобрать начальное приближение, что для всех остальных (последующих) существуют постоянные $C$ и $m$, что это соотношение выполнено, то $m$ называется скоростью сходимости метода. Если $m=1$, то метод сходится как геометрическая прогрессия и нам надо $C<1$. Если $m>1$, то на $C$ ограничений меньше, зато больше условий для начального приближения.

А что надо требовать от $\phi$, чтобы $m$ оказалось больше единички? Начнём
\[
  x_{n+1} - \ol x = \phi(x_n) - \phi(\ol x) = \phi\big(\ol x + (x_n - \ol x)\big) - \phi(\ol x) = 
  \phi(\ol x) + (x_n - \ol x)\phi'(\ol x) + \frac{(x_n-\ol x)^2}2\varphi'(\ol x) + \ldots - \phi(\ol x).
\]
Если $\phi'(\ol x)\ne 0$, то $m\approx 1$, $C\approx \phi'(\ol x)$. Согласуется с тем, что мы получали.

Если же у нас $\phi'(\ol x) = 0$, $\phi''(\ol x)\ne 0$. Тогда $m\approx 2$. Пишу нестрого, потому что потом будет целая теорема про метод второго порядка. Но набросок к действию у нас уже есть.

Попытаемся чуть по-другому действовать. Вот уравнение $f(x)=0$, вот корень $\ol x$, то есть $f(\ol x)=0$. Пусть $x_n\approx \ol x$. Представим его в виде корня уравнения, которое мы умеем решать. Самое простое, что мы умеем решить это линейное. Предположим, что у нас даже несколько точек $x_n$, $x_{n-1}$, \ldotst{} Через две точки простейшая идея провести секущую. Дальше следующая пара даёт секущую. На картинке получается очень хорошо.
% рисунок
Можно выписать общую формулу, это в общем несложно.
\[
  x_{n+1} = x_n - \frac{x-x_{n-1}}{f(x_n) - f(x_{n-1})} f(x_n).
\]
Для этого метода (при достаточно жётских ограничениях) $m = \frac{\sqrt{5} + 1}2\approx 1{,}62$.

Если проводить не секущие, а касательные, это называется методом Ньютона. 
%рисунок
Здесь формула
\[
  x_{n+1} = x_n - \frac{f(x_n)}{f'(x_n)}.
\]
Если корень простой кратности, то $m=2$.

Дальше возникает вопрос, а давайте теперь параболой функцию приближать. Пусть есть $x_{n-2},x_{n-1},x_n$.
Можно взять параболу через эти точки. Вещественных корней может и не быть. Ну возьмём какой-нибудь (по единому на весь алгоритм) комплексный корень. Следующий многочлен будет уже с комплексными коэффициентами. И работаем до тех пор, пока разность последних двух по модулю меньше $\e$. Такие программые есть. Не найдено многочлена, на котором программа бы сломалась (за лет пятьдесят), но и нет доказательства, что на всех многочленах работает.

Можно повернуть ситуацию чуть по-другому. Предположим то же самое. Есть три приближения, взять параболу. Она обязательно в одной точке пересечёт ось $y$. Можем повернуть ситуацию $x = x(y)$.  И такого сорта строим параболу $x = x(y)$. Ось иксов обязательно парабола пересечёт. Её и берём в качестве следующего приближений. Следующий шаг здесь всегда есть. А вот будет ли сходимость, это уже отдельный вопрос.

Можно продолжать приближать многочленами более высого порядка. Но это шаткий путь.

Вернёмся к методу Ньютона, который мы с вами обсудили
\[
  x_{n+1} = x_n - \frac{f(x_n)}{f'(x_n)}.
\]
Здесь $\phi(x) = x - \frac{f(x)}{f'(x)}$. Попробуем его прогнать через наши допущения. У нас пусть есть корень $f(\ol x) = 0$. Посчитаем
\[
  \phi'(\ol x) = 1 - \frac{\big(f'(\ol x)\big)^2 -f(\ol x)f''(\ol x)}{\big(f'(\ol x)\big)^2} = 0.
\]
У нас было рассуждение, что если производная ноль, то возможен метод порядка 2.

А теперь будет решать уравнение $f(x) = x^p$, $p\in \N,p\ge 2$.

Итак $x^p = 0$, 
\[
 x_{n+1} = x_n - \frac{x_n^p}{p x_n^{p-1}} = \left( \frac{p-1}{p} \right)x_n.
\]
Для нашего случая $m=1$, $C = \frac{p-1}{p}$. Итак чем выше кратность корня, тем метод работает хуже. Это нас наводит на мысль, что делать с кратными корнями. Есть несколько подходов. Первый излишне энергичный. Вместо уравнения $f(x) = 0$ решаем $\frac{f(x)}{f'(x)} = 0$. Если такой процесс нас не устраивает (большие производные или не умеем считать вторые производные).

Другой способ. Взяли мы $x_0$. Сгенерировали последовательность $\{x_n^{(1)}\} \to \ol x^{(1)}$. Дальше берём другое начальное приближение и прогоняем алгоритм для $\frac{f(x)}{x - 
\ol x^{(1)}} = 0$. Получаем $x_0^{(2)}$ и последовательность $\{x_n^{(2)}\} \to \ol x^{(2)}$. Дальше снова делим на разность $x-\ol x^{(2)}$. Если два подряд одинаковый, одному добавляем кратость. 

Дальше поговорим о системах. Единственный метод, который не зависит от размерности это метод Ньютона.

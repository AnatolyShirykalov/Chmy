
\section{Лекция 5}
У нас есть отрезок $[a,b]$. Есть функция $f(x)$, где $x\in[a,b]$. Есть натуральное $n$. $Q_n^0(x)$ "--- многочлен наилучшего равномерного приближения (МНРП) степени не выше $n$, если
\[
  \|f - Q_n^0\|_C\le \|f - Q_n\|_C,\pau \forall Q_n.
\]

У нас есть обозначение такое. Так как $Q_n^0$ (МНРП) существует, то $\|f - Q_n^0\| = \Delta_n(f)$.

Мы в прошлый раз сформулировали теорему о необходимом и достаточном условии. 

\begin{The}[Чебышёв]
  $Q_n$ "--- МНРП для $f\in C[a,b]$, если и только если
  \[
    (n+2)\colon x_0<x_1<\dots<x_{n+1},\ x_i\in[a,b],\quad
    f(x_i) - Q_n(x_0) = (-1)^i\alpha\cdot \|f - Q_n\|_C.
  \]
  Причём $\alpha=1$ или $-1$ сразу для всех $i$.
\end{The}
Точки $x_i$ называются точками чебышёвского альтернанса.
\begin{Proof}
  Докажем только справа налево. Слева направо муторно и неприятно. Непрерывность в основном нужна слева направо.

  \begin{The}[Валле-Пуссейна]\label{vp}
    Пусть у нас имеется некоторая функция $f(x)$ и многочлен степени $n$ $Q_n(x)$. Функция определена на $[a,b]$. Пусть существуют $n+2$ точки на $[a,b]$ $x_0<x_1<x_2<\dots<x_{n+1}$, такие, что 
    \[
      \sgn\big( f(x_i) - Q_n(x_i)\big)\cdot (-1)^i = \const.
    \]
    Тогда отсюда следует, что $\Delta_n(f):=\|f-Q_n^0\|\ge \min\limits_i\big|f(x_i)-Q_n(x_i)\big|=:\mu$.
  \end{The}
  Такое вот странное утверждение. Потом на самом деле из него всё будет следовать.
  \begin{proof}
    Если $\mu=0$, всё доказано. Пусть $\mu>0$. Рассмотрим знак вот такой вот штуки
    \[
      \sgn \Big(\big(f(x) - Q_n(x)\big) - \big(f(x) - Q_n^0(x_i)\big)\Big).
    \]
    Предположим, что $\Delta_n(f)<\mu$. Появился набор точек, в которых разность больше чем норма. Поставлю в нашем выражениии $x_i$
    \[
      \sgn \Big(\big(f(x_i) - Q_n(x_i)\big) - \big(f(x_i) - Q_n^0(x_i)\big)\Big)=
    \]
    Знак этого выражения определяется только следующей величиной
    \[
      \sgn\big( f(x_i) - Q_n(x_i)\big) = (-1)^i.
    \]
    Это по условию теоремы. Но при этом эта штука разность двух многочленов. Многочлен в $n+2$ точек имеет разные знаки, значит, $n+1$ корень. Следовательно это нулевой многочлен, то есть $Q_n(x) = Q_n^0(x)$. Значит, понятно, что есть неравенство 
    \[
      \Delta_n(f)\le \big|f(x_i) - Q_n(x_i)\big|.
    \]
    Вопрос: нужна ли в этой теореме непрерывность $f$.
  \end{proof}
  Доказываем справа налево теорему Чебышёва. Считаем, что у нас есть тот безумный набор $n+2$ точек. В каждой точке достигается норма и знак чередуется. Обозначим $\|f - Q_n\|_C = L$. В точках этого самого альтернанса мы получим, что
  \[
     \big| f(x_i) - Q_n(x_i)\big| = L \overset{\ref{vp}}= \mu \le \Delta_n(f)\le \|f - Q_n\| = L.
  \]
  Раз неравенство замкнулось, $Q_n$ это тоже МНРП.
\end{Proof}

Теперь я хочу с помощью этой теорему Чебышёва доказать что-то полезное, а именно единственность.
\begin{Ut}
  Есть функция $f\in C[a,b]$. Зафиксировали какое-то $n$. Значит, у нас есть $\Delta_n(f)$. Пусть существуют два многочлена $Q_n^k$ ($k=1,2$), на которых эта штука достигается$\colon \|f - Q_n^k\| = \Delta_n(f)$. 
  
  Тогда $Q_n^1= Q_n^k$.
\end{Ut}
\begin{Proof}
  Рассмотрим $\left\| f - \frac{Q_n^1 + Q_n^2}2\right\|\le \frac12 \|f- Q^1_n\| + \frac21 \|f - Q_n^2\| = \Delta_n(f)$. Меньше в этом неравенстве мы получить не можем. Значит, $\frac{Q_n^1+Q_n^2}2$ "--- МНРП, он не хуже остальных, работает теорема Чебышёва, существуют точки альтернанса.
  \[
    \left|\frac12 \big(Q_n^1(x_i) + Q_n^2(x_i)\big) - f(x_i)\right| = \Delta_n(f).
  \]
  Чуть-чуть пересоберём скобки
  \[
    \Big| \big(Q_n^1(x_i) - f(x_i)\big) + \big(Q_n^2(x_i) - f(x_i)\big)\Big| = 2\Delta_n.
   \]
   Значит, обе разности одного знака и равны по модулю $\Delta_n$. Таким образом, в $n+2$ точках многочлены $Q_n^1$ и $Q_n^2$ совпадают. Этого с избытком хватает, чтобы многочлены совпадали полностью.
\end{Proof}

Рисовать примеры не буду. Это вы сделаете на семинарах. Лучше кое-что ещё докажу.

Пусть $x\in[-1,1]$ и функция $f(x) = - f(-x)$, $Q_n^0$ "--- МНРП. Покажем, что $Q_n^0(-x) = -Q_n^0(x)$.

У нас $\forall\ x\in [-1,1]\pau \big|f(x)- Q_n^0(x)\big|\le \Delta_n(f) = \|f - Q_n^0\| = \min\limits_{Q_n}\|f-Q_n\|$. Подставим $f(-x)$, тоже должно работать
\[
  \big|f(-x) - Q_n^0(-x)\big|\le \Delta_n(f).
\]
Минус можно из $f$ вынести
\[
  \Big|f(x) - \big(-Q_n^0(-x)\big)\Big|\le \Delta_n(f).
\]
Значит, это тоже многочлен наилучшего приближения. А мы установили единственность.


Что ещё можно получить бесплатно. Пусть $f$ гладкая. По теореме Чебышёва разность $f(x) - Q_n^0$ меняет знак в $n+2$ точке. Значит, $\exists\ y_1,\dots,y_{n+1}\colon f(y_i) = Q_n^0(y_i)$. То есть наш многочлен наилучшего приближения является интерполяционным многочленом по $n+1$ точке. Мы можем воспользоваться старыми результатами
\[
  \|f - Q_n^0\|_C\le \| f - L_{n+1}\|_C\le \frac{\| f^{(n+1)}\|_C (b-a)^{n+1}}{(n+1)! 2^{2n + 1}}.
\]

Вот получили оценку сверху. Давайте закончим приближать многочленами.

\section{Быстрое дискретное преобразование Фурье}
Мне надо сейчас обозначить некий алгоритм. А воспользуемся этими знаниями мы только в мае.

Пусть у нас имеется некоторая функция периодическая с периодом единица, то есть $f(x+1)=f(x)$. Её в принципе можно разложить в ряд Фурье, то есть написать вот так
\[
  f(x) = \rY k{-\infty} a_k \exp(2\pi ik x),\qquad \sum\limits_k|a_k|<\infty.
\]
Зафиксируем $N>0$. И будем рассматривать точки $x_l = \frac lN$, где $k\in \Z$. Для дальшейшего, чтобы не писать $f(x_l)$, обозначим $f(x_l) = f_l$. Тогда из представления ряда Фурье совершенно грандиозным образом можно привести подобные слагаемые.

Пусть $k_2 - k_1 = kN$. Тогда $k_2x_l - k_1 x_l = k\frac{ Nl}N = kl$. Соответственно
\[
  \exp(2\pi i k_1 x_l) = \exp (2\pi i k_2x_l).
\]
Таким образом,
\[
  f_l = \RY k0{N-1} A_k\exp(2\pi i k x_l),\qquad A_k = \rY p{-\infty} a_{k+pN}.
\]
Получили конечное разложение, однако сами коэффициенты $A_k$ "--- это, конечно, целая история.

Вопрос, если есть значения функции в узлах, можем ли мы посчитать $A_k$ и как не считать бесконечную сумму. И обратный вопрос: как, зная $A_k$, восстановить исходную функцию $f$.
Я расскажу несколько приёмов.

Пусть есть отрезок $[0,1]$. И у нас есть такая сеточка $x_k = \frac kN$, $k=0,\dots, N-1$. Будем рассматривать всевозможные сеточные функции, то есть определённые на такой сеточке. Введём скалярное произведение
\[
  (f,g) = \frac1 N \RY k0{N-1} f_k \ol g_k.
\]
Для этого скалярного произведения существует ортонормированная система функций. Давайте эту систему предъявим.
\[
  g^k(x_l) = \exp(2\pi i k x_l), \quad 0\le k< N.
\]
давайте покажем, что эта система функций "--- то, что нам надо. Возьмём две функции и скалярно перемножим.
\[
  (g^k,g^j) = \frac1N \RY l0{N-1}\exp\left(2\pi i\frac{k-j}Nl\right).
\]
Если $k=j$, то каждая $\exp$ даёт единичку, всего и $N$, в сумме дают единичку.
Пусть $k\ne j$. Тогда перед нами геометрическая прогрессия
\[
  (g^k,g^j) = \frac1N \frac{\exp\big(2\pi i(k-j)\big) -1}{\exp\left(2\pi i\frac{(k-j)}N\right) - 1} = 0.
\]

Из этого всего мы сейчас получим прямое и обратное преобразования Фурье.
\[
  f_l = \RY k0{N-1} A_k \exp(2\pi i k x_l) = \RY k0{N-1} A_k g^k(x_l).
\]
Это по сути и есть обратное преобразование Фурье: знаем $A_k$, восстанавливаем Функцию. Теперь хотим прямое. Умножим нашу функцию скалярно на $g^j$.
\[
  A_j = (f,g^j) = \frac1N\RY l0{N-1} f_l\exp\left(2\pi i j x_l\right).
\]
На следующей лекции мы разберём алгоритм, который позволяет это преобразование делать быстро.
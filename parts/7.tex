\section{лекция}

В прошлый раз мы с вами приступили к численному дифференцированию. Остановились на правиле Рунге оценки погрешности. Сейчас на примере вычисления первой производной опишу алогитм «метод Ромберга». Вещь несколько устаревшая. Но тем не менее добавляет представления, о том, что происходит с ошибками при вычислении и в чём специфика нашей системы счисления.

Итак на надо для $\e>0$ получить $f'(x)$. $\e$ берём на пределе наших вычислительных возможностей. Давайте посмотрим, что можно выжать из тех выкладок, которые мы сделали на прошлой лекции.

Вот что у нас есть в активе
\[
  f'(x) = \frac{f(x+h) -f(x-h}{2h} + c h^2 + O(h^4).
\]
Поскольку точки выбрали симметрично, получили не $h^3$, а $h^4$.
На самом деле тут
\[
  f'(x) = \underbrace{\frac{f(x+h) -f(x-h}{2h}}{D_{(2)}^{(1)}(h)f(x)} + c h^2 + c_1 h^4 + c^2 h^6+ \dots.
\]

Будем потихоньку избавляться от этих членов.
  \begin{eqnarray*}
f'(x) &=&  D_{(2)}^{(1)}(h)f(x) + ch^2 + O(h^4);\\
f'(x) &=& D_{(2)}^{(1)}(h/2)f(x) + c\left( \frac h2 \right)^2 + O(h^4);\\
 ch^2 &=&  \frac{D_{(2)}^{(1)}\left( \frac h2 \right)f(x) - D_{(2)}^{(1)}(h)f(x)}{1 - \frac14} + O(h^4).
\end{eqnarray*}

Это ещё правило Рунге. Что мы можем получить из этого
\[
  f'(x) = D_{(2)}^{(1)}(h)f(x) + \frac{D_{(2)}^{(1)}\left( \frac h2 \right)f(x) - D_{(2)}^{(1)}(h) f(x)}{1 - \frac h4} + O(h^4)=
  D_{(4)}^{(1)}\left( \frac h2 \right)f(x) + c_1 h^4 + O(h^6).
\]
Дальше можно снова применять правило Рунге и обирать следующие члены.

Из шага $h$ мы получаем формулу $D_{(2)}^{(1)}(h)$. Для двух шагов $h$ и $h/2$ мы можем построить формулу $D_{(4)}^{(1)}$ четвёртого порядка, если нас не устроила погрешность. Если и она нас не устроила, мы можем построить формулу $D_{(6)}^{(1)}$ шестого порядка из трёх шагов $h$, $h/2$, $h/4$.

Если бы мы пользовались только одним правилом Рунге, то есть выбирали только $D_{(2)}^{(1)}\left( \frac h{2^n} \right)$, мы бы всё время получали формулу второго порядка. А мы предъявили алгоритм получения последовательности формул с возрастающим порядком точности.

Всё это будет давать формулы вида $\frac 1{h} \Sigma c_i f_i.$

\section{Численное интегрирование}
Задача более широкая. Поставим задачу, что мы хотим вычислить. Хотим получить в результате число. Таким образом, интегралы у нас будут определённые. Конечно, они будут собственные или несобственные. В одномерном случае у нас задачи будут сводиться к постановке
\[
  \int\limits_a^b f(x)\,dx.
\]
Если несобственный, мы будем отрезкать хвост от промежутка интегрирования с какой-то точностью.
\[
  \int\limits_a^b f(x)\,dx \approx \RY i1n c_i f(x_i).
\]
Числа $c_i$ принято называть коэффициентами, $x_i$ "--- узлами.

Принято на ряду с этим рассматривать вот такое обобщение
\[
  \int\limits_a^b f(x) p(x)\,dx\approx \RY i1n c_i f_i.
\]
При таком подходе $p(x)$ называется весовой функцией, $c_i$ будут зависеть от $f(x)$.

Как мы увидим, погрешность, как бы мы её ни определяли, будет зависеть от гладкости функции. Если гладкости нет, то оценки погрешности у нас будет неверные. В функцию $p(x)$ мы все нехорошести занесём, и при соблюдении особых правил, погрешность будет зависеть только от гладкости $f(x)$.

Пока о никаких весовых функциях говорить не будем.

Будем обозначать $I(f) = \int\limits_a^b f(x)\,dx$, когда будет понятно, какие пределы интегрирования. Также будем обозначать $S(f) = \RY i1n c_i f(x_i)$.

Погрешность будем обозначать, $R(f)$, то есть
\[
  I(f) = S(f) + R(f).
\]

Если нам потребуется вычислять двумерный интеграл или более. Смысл будет тот же, только формулы не будут уже называться квадратурными.

Как мы будем подбирать узлы и коэффициенты наших формул?
\subsection{Квадратурные формулы Ньютона"--~Котеса}
Узлы выбираются равномерно по отрезку. Если $n=1$, то $x_1 = \frac{a+b}2$. Если $n\ge2$, то
\[
  x_1 = a,\quad x_n = b,\quad x_{i+1} - x_i = h = \const.
\]

Подход для вычисления коэффициентов очень приятный. После того, как мы зафиксировали $x_1,\dots,x_n$. Мы можем вспомнить, что у нас было при интерполировании функции многочленом Лагранжа.
\[
  f(x) = L_n(x) +\frac{\omega_n(x) f^{(n)}(\xi)}{n!},\quad
\xi\in[a,b],\pau \omega_n(x) = \prod\limits_{i=1}^n (x-x_i).
\]
Что такое $L_n$ мы тоже знаем
\[
  L_n(x) = \RY i1n f(x_i) \prod\limits_{j\ne i} \frac{x-x_j}{x_i-x_j} = 
  \RY i1n f(x_i)\Phi_i(x).
\]

Подставим в основную формулу 
\[
  I(f) = (b-a)\RY i1n c_i f(x_i) + R(f),
\]
где
\[
  c_i = \frac1{b-a} \int\limits_a^b \Phi_i(x)\,dx,\qquad
  R(f) = \frac1{n!} \int\limits_a^b \omega_n(x) f{(n)}\big(\xi(x)\big)\,dx.
\]
Оказывается, что при таком подходе $c_i$ не будут зависеть от $a$ и $b$, только от $n$. Это хорошая часть. Что есть плохое: оценка для погрешности. Она, скажем прямо, непрактичная. Стоит $n$-я проивзодная не пойми в какой точке, ещё интеграл брать.

Давайте сейчас, ну во-первых, я сформулирую некоторое утверждение в виде задачи.
\begin{Ut}
  Понятно, что точки расположены равномерно по отрезку не просто так. Оказывается, что коэффициенты $c_i$ в нашей формуле обладают определённой симметрией. Если их количество нечётно, один узел попадают в середину. Все расположены симметрично относительно середеины. Утверждение в том, что $c_i = c_{n+1-i}$.
\end{Ut}
\begin{Proof}
Намечу план доказатепльства. Надо здесь всё обезразмерить. Можно сделать замену $x= a+ t h$.
И при вычислении интегралов сделать эту замену и в лоб выписать, чему равняются $c_i$. Не очень приятная задача, но очень полезно это сделать.
\end{Proof}

Мы приняли тот факт, что $c_i = c_{n+1-i}$. И ещё одно свойство, самое приятное: если функцию взять константу, погрешность интерполяции будет ноль. А значит, и  погрешность нашей формулы ноль. Следовательно, $c_1+\dots c_n = 1$.

Нам надо договориться, о каком-то критерии эффективности. Будем подставлять многочлены и требовать, чтобы формула была точна. Наибольшая степень многочлена, для которого формула точна, будет порядком точности. Есть ещё малый параметр: расстояние между узлами, через который можно оценивать погрешность. Но как ни смешно это, удонее оценивать через длину отрезка $[a,b]$.

Начнём с многочленов. Какая максимальная степень многочленов, для которых формула точна. Оказывается, если $n$ нечётно, то формула точна для многочленов степени $n$.

Имеется у нас равномерная сетка и интеграл мы замеряем, как договорились
\[
  \int\limits_a^b f(x)\,dx\approx (b-a) \RY i1n c_i f(x_i),\qquad n= 2k +1.
\]
Рассмотрим произвольный многочлен степени $n$, обозначим
\[
  Q_n(x) = a_n x^n + \dots.
\]
Давайте через $\ol x$ обозначим $\ol x = \frac{a+b}2$. Тогда
\[
  Q_n(x) = a_n(x-\ol x)^n + P_{n-1}(x).
\]

Если мы затеем вычислять наш интеграл от первого слагаемого, так как оно нечётно,
\[
  I\big(a_n(x-\ol x)^2\big) = 0.
\]
При этом $S\big(a_n(x\ol x)^n\big) = 0$. Значит,
\[
  I(Q_n) = I\big(P_{n-1}(x)\big) = S\big(P_{n-1}(x)\big) = S(Q_n).
\]

Теперь по поводу малого параметра. Вот подумаем. Что мы можем сказать о погрешности? Есть у нас $n$-я производная и многочлен. Во всех формулах у нас получалось: чем больше отрезок, на котором происходит интерполяция, тем оценка погрешности хуже. Мы не можем сказать, что интегрировать будем только по маленьким отрезкам. Так как интеграл есть аддитивный функционал. Мы его разобьём на сумму интегралов по элементарным отрезкам.
\[
  I(f) = \int\limits_a^b f(x)\,dx = \RY j1{N-1} I_j(f) = \RY j1{N-1} \int\limits_{x_j}^{x_{j+1}} f(x)\,dx.
\]
 При этом обозначим $x_{j+1}-x_j = h$. Тогда если на каждом элементарном отрезке погрешность порядка $O(h^p)$, то на всём $[a,b]$ будет ошибка $\RY j1{N-1} O(h^p) = O(h^{p-1})$.

Начнём с простых ситуаций.
Пусть у нас $n=1$. Мы обозначаем $h = b-a$.
\[
  I(f) = \int\limits_a^b \approx = h c_1 f(\ol x).
\]
Коэффициент $c_1$ подбирается так, чтобы на многочленах до первой степени была точная формула.

Пусть $f\equiv 1$. Тогда $c_1=1$. Значит,
\[
  I(f) = h f(\ol x) = \Pi(f).
\]
Это называется формула прямоугольника. Ясно, что в линейном случае справа и слева в разные стороны будет торчать треугольник.

Пусть теперь $n=2$. Тогда
\[
  I(f) = h\big(c_1 f(a) - c_2 f(b)\big).
\]
Спокойно сюда подставляем $f=1$ и $f=x$. После определённых усилий получим, что $c_1 = c_2 = \frac12$.
Формулу
\[
  I(f) = h\left( \frac{f(a) + f(b)}{2} \right) = T(f)
\]
традиционно принято называть формулой трапеции.

Теперь нам хотелось бы получить разложение ошибки по $h$.

Вопрос маленький. Чему равен
\[
  \int\limits_a^b (x-\ol x)^j = \begin{cases}
  h,&j=0;\\
  0,&j=1;\\
  \frac{h^3}{12},&j=2;\\
  0,&j=3;\\
  \frac{h^5}{80},&j=4.
\end{cases}
\]
Этого мне будет достаточно.

Разложим нашу функцию в ряд относительно середины отрезка
\[
  f(x) = f(\ol x) + (x-\ol x) f'(\ol x) + \frac{(x-\ol x)^2}2f''(\ol x) + \dots
\]
И подставим это разложение в наш интеграл
\[
  \int\limits_a^b f(x)\,dx = \underbrace{h f(\ol x)}_{\Pi(f)} + \underbrace{\frac{h^3}{24} f''(\ol x) + O(h^5)}_{\R_\Pi}.
\]
Таким образом, $\big|I(f) - \Pi(f)\big| = O(h^3) = \frac{h^3}{24} f''(\ol x) + O(h^5)$. На самом деле мы фактически показали, как выписать весь ряд погрешностей.

С формулой трапеции немножко больше потрудиться надо. Подставим в то же разложение $f$ относительно $\ol x$ значения в концах отрезка.
\begin{eqnarray*}
f(a) &=& f(\ol x) - \frac h2 f'(\ol x) + \frac{h^2}8 f''(\ol x) - \frac{h^3}{48} f'''(\ol x) + \frac{h^4}{384} f^{(4)}(\ol x)+\dots\\
f(b) &=& f(\ol x) + \frac h2 f'(\ol x) + \frac{h^2}8 f''(\ol x) + \frac{h^3}{48} f'''(\ol x) + \frac{h^4}{384} f^{(4)}(\ol x)+\dots\\
\end{eqnarray*}
Давайте их сложим и умножим на $h$.
\begin{equation}\label{eq7-1}
  h f(\ol x) =  h\frac{f(a) + f(b)}{2} - \frac{h^3}8f''(\ol x) - \frac{h^5}{384} f^{(4)}(\ol x) + \dots
\end{equation}

Подставим это в формулу для треугольника, которая выглядела так
\[
  I(f) = hf(\ol x) + \frac{h^3}{24} f''(\ol x) + \dots = \frac h2 \big(f(a) = f(b)\big) - \frac{h^3}{12} f''(x) - \frac{h^5}{480} f^{(4)}(\ol x) + \dots.
\]
По модулю получилось в два раза хуже. Но главное, конечно, порядок.

Давайте попробуем на основе этих двух формул построить формулу третьего порядка.
\begin{eqnarray*}
I(f) - \Pi(f) &=&  \frac{h^3}{24} f''(\ol x) + \frac{h^5}{1\,920} f^{(4)}(\ol x) +\dots;\\
I(f) - T(f) &=&  -\frac{h^3}{12} f''(\ol x) - \frac{h^5}{480} f^{(4)}(\ol x) + \dots.\\
\end{eqnarray*}
Первое умножим на два, сложим со вторым, и всё поделим на три
\[
  I(f) = \frac23 \Pi(f) + \frac13 T(f) - \frac{h^5}{2\,880}f^{(4)}(\ol x) + \dots = C(f) + O(h^5).
\]
Это называется формулой Симпсона. Узлы здесь задействованы такие: $a,b,\ol x$. Это формула из нашего семейства. Точна для многочленов до третьей степени включительно. И мы получили даже её общий вид
\[
  C(f) = \frac h6\big(f(a) + 4 f\left( \frac{a+b}2 \right) + f(b).
\]

В качестве хорошей задачи можно взять и посчитать интерполяционный многочлен по трём точкам и убедиться, что получается такая же формула.
